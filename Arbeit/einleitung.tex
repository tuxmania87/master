%einleitung

SQL (structured query language) ist eine Datenbanksprache, die in relationalen Datenbanken zum Definieren, Ändern und Abfragen von Datenbeständen benutzt wird. Basierend auf relationaler Algebra und dem Tupelkalkül, ist sie einfach aufgebaut und ähnelt der englischen Sprache sehr, was Anfragen deutlich verständlicher gestaltet. SQL ist der Standard in der Industrie wenn es um Datenbankmanagementsysteme (DBMS) geht. Zu bekannten Vertretern gehören Oracle Database von Oracle, DB2 von IBM, PostgreSQL von der PostgreSQL Global Development Group, MySQL von der Oracle Corporation und SQLServer von Microsoft.

Die Umsetzung von SQL als quasi-natürliche Sprache erlaubt es Anfragen so zu formulieren, dass sie allein mit dem Verständnis der natürlichen Sprache verständlich sind. Dieser Umstand hat auch dazu geführt, dass heutzutage relationale Datenbanksysteme mit SQL beliebt sind und häufig eingesetzt werden. 
Dies führt allerdings auch dazu, dass es mehrere -- syntaktisch unterschiedliche -- Anfragen geben kann, welche semantisch identisch sind. Manche sehen sich dabei ähnlich, andere Anfragen kann man nur nach Umformen oder umschreiben ineinander überführen. 

\section{Motivation}

Ein gängiges Mittel um herauszufinden ob zwei SQL-Anfragen das gleiche Ergebnis liefern, ist es die Anfragen auf einer Datenbank mit vorhandenen Daten auszuführen. Dies bildet jedoch lediglich Indizien für eine mögliche semantische Gleichheit. Da man die zwei zu vergleichenden Anfragen nur auf einer endlichen Menge von Datenbankzuständen testen kann, ist nie ausgeschlossen, dass nicht doch ein Zustand existiert, der unterschiedliche Ergebnisse liefert. Weiterhin stehen solche Testdaten nur im begrenzten Umfang zur Verfügung oder Daten müssten händisch eingetragen werden oder von freien Internetdatenbanken beschafft werden. Dies kostet Zeit und Arbeitskraft, welche im universitärem Umfeld meist beschränkt ist. So haben Hochschulen immer weniger Geld für Tutoren oder Hilfskräfte, was die Zeit der wenigen Mitarbeiter umso wertvoller macht.

Durch diese Situation sind Professoren immer öfter dazu gezwungen mehr Lehre und weniger Forschung zu betreiben, was aber offensichtlich auch keine gute Lösung ist. Häufig werden dem Lernenden Übungsaufgaben gestellt, die dieser dann innerhalb einer Frist bearbeitet und abgibt. Diese müssen dann kontrolliert und wieder ausgehändigt werden. Bei diesem Prozess kann nur schwer auf die einzelnen Fehler der Studenten eingegangen werden. Auch ist es ein zusätzlicher Zeitaufwand herauszufinden, welche Fehler besonders häufig auftreten. Des weiteren sind manche Lernende auch gewillt mehr zu üben um sich gerüstet für eine Klausur zu fühlen oder Lernende möchten gezielt ein Thema üben, welches sie noch nicht gut beherrschen. All das ist mit Übungsaufgaben und Übungen innerhalb der Hochschule schwer zu erreichen. 

Das Programm, welches im Rahmen dieser Arbeit entwickelt wird, soll helfen all diese Probleme zu lösen. Es soll mit wenig Aufwand möglich sein für den Mitarbeiter der Universität neue Aufgaben in das System einzupflegen. Durch Abspeicherung sämtlicher Lösungsversuche des Lernenden können einzelne Aufgaben vom Dozenten durch das Programm auf häufig auftretende Fehler untersucht werden. Damit kann in der Übung gezielt besprochen werden, was noch oft falsch gemacht wird.

\section{Aufgabenstellung}

Nach der theoretischen Ausarbeitung soll ein Programm entwickelt werden, welches in der Lage ist zwei SQL-Anfragen zu vergleichen. Dieser Vergleich soll, im ersten Schritt, lediglich mit Musterlösung, Lösung des Lernenden und Datenbankschema möglich sein. Im zweiten Schritt werden die zwei Anfragen auch gegen reale Daten geprüft. Da die Fehlermeldung des Standardparser von SQL sehr allgemein gehalten sind, ist es auch wünschenswert, dass das Programm konkretere Hinweis- und Fehlermeldung ausgibt, als es der Standard SQL-Parser vermag. Weiterhin sollen die Fehlermeldung sich konkret auf den Unterschied von Musterlösung zur Lösung des Lernenden beziehen. Damit das Programm möglichst plattformunabhängig bedient werden kann, soll es als Webseite auf einem Server zur Verfügung gestellt werden. Da als Programmiersprache Java gewählt wurde, bieten sich die JSP (java server pages), sowie java-servlets als Umsetzung dieser Anforderung an.

Ein mögliches Haupteinsatzgebiet ist die Lehre, so wie die Untersuchung des Lernfortschritts von Studenten oder anderen Interessierten, die den Einsatz von SQL erlernen möchten. So kann das Programm dem Lernenden nicht nur sinnvolle Hinweise bei einer falschen Lösung geben, sondern auch erläutern, ob die gefundene Lösung eventuell zu kompliziert gedacht war. Des weiteren ist es aufgrund der zentralisierten Serverstruktur möglich, Lösungsversuche des Lernenden zu speichern und eine persönliche Lernerfolgskurve anzeigen zu lassen. Damit hätten Studenten und Lehrkräfte die Möglichkeiten Lernfortschritte zu beobachten und Problemfelder (etwa JOINs) zu erkennen um diese dann gezielt zu Bearbeiten. Dozenten könnten so im Zuge der Vorbereitung der Übung oder Vorlesung sich die am häufigsten aufgetretenen Fehler anzeigen lassen um diese dann mit den Studenten direkt zu besprechen.

Damit es ist möglich eine Lernplattform aufzubauen, die dem Studenten mehrere Auswertungsinformationen über seinen Lernerfolg und seine Lösung deutlich macht. So kann die Lehrkraft eine Aufgabe mit samt Musterlösung und Datenbankschema hinterlegen und der Student kann daraufhin seine Lösungsversuche in das System eintragen. Durch sinnvolles Feedback ist es ihm so möglich beim Üben direkt zu lernen. Weiterhin kann man eine solche Plattform auch für Tutorien oder Nachhilfe überall da benutzen, wo SQL gelernt wird. Vorteile hier wären, dass man mehrere verschiedene Aufgaben stellen kann ohne viel Zeit beim Einpflegen von neuen Aufgaben verbringen zu müssen.

Weitere Einsatzgebiete könnten im sich im Unternehmen befinden. So könnte man bei einer geplanten Umstrukturierung oder Erzeugung von Datenbanken bereits Anfragen prüfen und vergleiche bevor man sich u.U. teure Testdaten kauft oder Daten migrieren muss.

\section{Aufbau der Arbeit}

Im aktuellen Kapitel haben wir geklärt warum es eine Notwendigkeit für das Thema SQL-Vergleich gibt. Zu dem wurde geklärt, was das Ziel der Arbeit ist. More tt.

Im Kapitel \ref{chap:forschung} betrachten wir den aktuellen Forschungsstand zur Thematik Lernplattformen und SQL. Das Ergebnis dieser Arbeit soll ein Produkt sein, was hauptsächlich in der Lehre eingesetzt wird. Daher ist es wichtig bereits vorhandene Lernplattformen zu untersuchen. Dabei interessieren uns insbesondere Gemeinsamkeiten und Unterschiede zu unserer Arbeit. Wir werden feststellen, dass jede Plattform auf eine Feinheit spezialisiert ist und andere Punkte dann eine untergeordnete Rolle spielen. Weiterhin ist diese Bestandsaufnahme wichtig, da sie uns aufzeigen kann, wie man mögliche Ansätze miteinander verknüpft. Dieser Gedanke wird im Kapitel \ref{chap:ausblick} genauer erläutert.

Das Problem ``Sind zwei SQL-Anfragen äquivalent'' ist nicht entscheidbar. Aus diesem Grund klären wir im Kapitel \ref{chap:theorie} wie wir den Entscheidungsprozess angehen wollen, so dass zumindest eine Teilmenge von SQL-Anfragen bearbeitet werden kann. Wir klären also zunächst, wie unser Programm vorgehen wird. Danach werden die einzelnen Schritte, die zum Vergleich notwendig sind, erklärt und besprochen. Dabei diskutieren wir für einzelne Teilschritte auch mehrere Herangehensweisen. Weiterhin werden wir alle möglichen Analyseschritte theoretisch untermauern, auch wenn nicht alle davon im Programm umgesetzt werden können. Mehr dazu im Abschnitt >>Grenzen des Parsers<<. Alle später umgesetzten Algorithmen werden in diesem Kapitel vorgstellt, erarbeitet und diskutiert.


Im nachfolgenden Kapitel \ref{chap:software} beschreiben wir die verwendete Software. Neben der üblichen Beschreibung der verwendeten Software, wird insbesondere auf den verwendeten Parser und die Java-Servelets eingegangen. Zu klären ist hier wie genau der Parser funktioniert und was er nicht kann. Daraus leitet sich eine gewisse Beschränkung in der praktischen Umsetzung ab. Da wir im vorherigen Kapitel allerdings alle theoretischen Betrachtungen ausführlich erläutert haben, stellt es kaum ein Problem dar, die vorgestellten Algorithmen auf einen anderen Parser zu übertragen. Ein weiterer Aspekt dieses Kapitels ist es, dem Leser klar zu machen wie Java-Servelets funktionieren und wie genau wir sie für unser Programm einsetzen.

In Kapitel \ref{chap:praxis} wird der Aufbau des Programms geklärt und erläutert. Dabei gehen wir den strukturellen Aufbau durch und klären, wie einzelne Aspekte aus Kapitel \ref{chap:theorie} umgesetzt werden konnten. Weiterhin klären wir, was Aufgrund gewisser Beschränkungen nicht umsetzbar war. Wir diskutieren die Struktur des Programms hier eingehend auf Konzepte der Softwaretechnik, wie z.B.: Wartbarkeit, Erweiterbarkeit usw.

TODO: Ergebnisse und Ausblick (zusammenfassen)


\section{Produkt}

Es wurde im Rahmen der Aufgabenstellung ein Programm entwickelt, welches es erlaubt zwei SQL-Anfragen miteinander zu vergleichen. Dazu wurde eine Lernplattform auf Basis von Java-Servlets geschaffen. Der Lernende meldet sich an der Plattform an und wählt eine Kategorie aus. Nun wird ihm eine Sachaufgabe gestellt und ein Datenbankschema angezeigt. Er soll nun daraus eine SQL-Anfrage formulieren, die die Aufgabenstellung löst. Dabei bekommt der Lernende Feedback vom Programm. Dies schließt sowohl Hinweise als auch konkrete Fehlermeldungen ein. Hat der Lernende die Aufgabe bereits mehrfach bearbeitet, so kann er sich seine vorherigen, eingesandten Lösungen anschauen und seinen Lernerfolg leicht verfolgen. Das Programm zeigt auch an, in welcher Kategorie der Lernende noch große Defizite hat.

Der Dozent hat das Programm vorher einmalig mit einer Reihe von Aufgaben bestückt. Dazu gibt der Dozent eine textuelle Beschreibung der Aufgabe, eine oder mehrere SQL-Anfragen als Musterlösungen, ein Datenbankschema und optional eine Datenbank an, auf der Beispieldaten vorhanden sind. 

Das Programm läuft im Wesentlichen in zwei Schritten ab. Im ersten Schritt versucht es, die zwei Anfragen miteinander zu vergleichen ohne den Einsatz von externen Daten. Gelingt dies, ist gezeigt, dass die Lösung des Studenten mit der Musterlösung übereinstimmt. Das Programm meldet Erfolg und zeigt eventuell abweichende Komplexitätsmaße an. Mehr dazu im Kapitel [Komplexitätsmaße].

Schlägt der erste Schritt fehl, wird die Anfrage des Lernenden auf der angegebene Datenbank verarbeitet und mit den Ergebnistupeln verglichen, die die Musterlösung liefert. Sind beide in allen Beispielzuständen gleich, so wird dem Dozenten gemeldet, dass eine eventuelle neue Musterlösung gefunden wurde, die strukturell so unterschiedlich ist, dass sie nicht auf die bisherige Musterlösung angepasst werden konnte. Wir können in einem solchen Fall nicht mit Sicherheit sagen, ob die Lösung falsch oder richtig ist, da dieses Problem im Allgemeinen nicht entscheidbar ist. Daher muss ein Mensch -- in Form des Dozenten -- solche Lösungen noch einmal prüfen.

Schlägt aber auch der zweite Schritt fehl, so können wir sicher sein, dass die Lösung des Lernenden falsch ist. Das Programm meldet dann eine Fehlermeldung so wie mögliche Hinweise, was der Lernende falsch gemacht haben könnte.


