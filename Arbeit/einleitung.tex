%einleitung
SQL (structured query language) ist eine Datenbanksprache, die in relationalen Datenbanken zum Definieren, Ändern und Abfragen von Datenbeständen benutzt wird. Basierend auf relationaler Algebra, ist sie einfach aufgebaut und ähnelt der englischen Sprache sehr, was Anfragen deutlich verständlicher gestaltet. Durch Standardisierungen in verschiedenen Versionen ist es möglich gewesen, dass mehrere DBMS (database management system) auf der Basis von SQL erstellt wurden. Zu bekannten Vertretern gehören Oracle Database von Oracle, DB2 von IBM, PostgreSQL von der PostgreSQL Global Development Group, MySQL von der Oracle Corporation und MS-SQL von Microsoft.

Die Umsetzung von SQL als quasi-natürliche Sprache erlaubt es Anfragen so zu formulieren, dass sie allein mit dem Verständnis der natürlichen Sprache verständlich sind. Dieser Umstand hat auch dazu geführt, dass heutzutage relationale Datenbanksysteme mit SQL beliebt sind und häufig eingesetzt werden. 
Dies führt allerdings auch dazu, dass es mehrere -- syntaktisch unterschiedliche -- Anfragen geben kann, welche semantisch identisch sind. Manche sehen sich dabei dennoch ähnlich andere gleiche Anfragen kann man nur nach Umformen oder umschreiben ineinander überführen. 

Ein gängiges Mittel um herauszufinden ob zwei SQL-Anfragen das gleiche Ergebnis liefern, ist es die Anfragen auf einer Datenbank mit vorhandenen Daten auszuführen. Dies bildet jedoch lediglich Indizien für eine mögliche semantische Gleichheit. Da man die zwei zu vergleichenden Anfragen nur auf einer endlichen Menge von Datenbankzuständen testen kann, ist nie ausgeschlossen, dass nicht doch ein Zustand existiert, der unterschiedliche Ergebnisse liefert. Weiterhin stehen solche Testdaten nur im begrenzten Umfang zur Verfügung oder Daten müssten teuer besorgt werden, was nur für den Vergleich von zwei Anfragen möglicherweise zu teuer sein kann.

Diese Arbeit versucht daher zunächst die theoretische Grundlage und Analyse zu liefern, wie es möglich ist zwei Anfragen miteinander zu vergleichen, wenn man keine Datensätze zur Verfügung hat, sondern lediglich Kenntnis über die Datenbankstruktur, also das Datenbankschema, besitzt.

Nach der theoretischen Ausarbeitung soll ein Programm entwickelt werden, welches in der Lage ist zwei SQL-Anfragen mit erarbeiteten Strategien, zu vergleichen. Da die Fehlermeldung des Standardparser von SQL sehr allgemein gehalten sind, ist es auch wünschenswert, dass das Programm konkretere Hinweis- und Fehlermeldung ausgibt, als es der Standard SQL-Parser vermag. Damit das Programm möglichst Plattformunabhängig bedient werden kann, soll es als Webseite auf einem Server zur Verfügung gestellt werden. Da als Programmiersprache Java gewählt wurde, bieten sich die JSP (java server pages), sowie java-servlets als Umsetzung dieser Anforderung an.

Ein mögliches Haupteinsatzgebiet ist die Lehre, so wie die Untersuchung des Lernfortschritts von Studenten oder anderen Interessierten, die den Einsatz von SQL erlernen möchten. So kann das Programm dem Lehrling(?) nicht nur sinnvolle Hinweise bei einer falschen Lösung geben, sondern auch erläutern, ob die gefundene Lösung eventuell zu kompliziert gedacht war. Des weiteren ist es aufgrund der zentralisierten Serverstruktur möglich, Lösungsversuche des Lernenden zu speichern und eine persönliche Lernerfolgskurve anzeigen zu lassen. Damit hätten Studenten und Lehrkräfte die Möglichkeiten Lernfortschritte zu beobachten und Problemfelder (etwa JOINs) zu erkennen um diese dann gezielt zu Bearbeiten.

Damit es ist möglich eine Lernplattform aufzubauen, die dem Studenten mehrere Auswertungsinformationen über seinen Lernerfolg und seine Lösung deutlich macht. So kann die Lehrkraft eine Aufgabe mit samt Musterlösung und Datenbankschema hinterlegen und der Student kann daraufhin seine Lösungsversuche in das System eintragen. Durch sinnvolles Feedback ist es ihm so möglich beim Üben direkt zu lernen. Weiterhin kann man eine solche Plattform auch für Tutorien oder Nachhilfe überall da benutzen, wo SQL gelernt wird. Vorteile hier wären, dass man mehrere verschiedene Aufgaben stellen kann ohne sich jedes mal um Testdaten zu kümmern.

Weitere Einsatzgebiete könnten im sich im Unternehmen befinden. So könnte man bei einer geplanten Umstrukturierung oder Erzeugung von Datenbanken bereits Anfragen prüfen und vergleiche bevor man sich u.U. teure Testdaten kauft oder Daten migrieren muss.
