%\documentclass[11pt]{scrartcl}


%ronny
\documentclass[12pt]{scrreprt}

\usepackage{primaer}
\usepackage[backend=bibtex]{biblatex}
\usepackage[babel,german=quotes]{csquotes}
\usepackage{setspace}
\usepackage{subcaption}
\usepackage{txfonts}
\usepackage{algorithm}
\usepackage[ngerman]{babel}
%\usepackage{algorithm}
\usepackage{algorithmic}
\usepackage{tikz}
\usepackage{hyperref}
\usepackage{listings}
\usetikzlibrary{trees}
\usetikzlibrary{shapes}
\usetikzlibrary{automata}
\usetikzlibrary{backgrounds}
\usetikzlibrary{plotmarks}
\usetikzlibrary{arrows}
%ende ronny

\bibliography{literatur}
\defbibheading{bibliography}[\bibname]{%
\chapter{#1}%
\markboth{#1}{#1}}

\AtEveryBibitem{% Clean up the bibtex rather than editing it
 \clearfield{doi}
 \clearfield{isbn}
 \clearfield{issn}
 %\clearlist{location}

 
  \ifboolexpr
    {
      test { \ifentrytype{article} }
      or
      test { \ifentrytype{inproceedings} }
    }
    {\clearfield{url}}
    {}%
}

\definecolor{javared}{rgb}{0.6,0,0} % for strings
\definecolor{javagreen}{rgb}{0.25,0.5,0.35} % comments
\definecolor{javapurple}{rgb}{0.5,0,0.35} % keywords
\definecolor{javadocblue}{rgb}{0.25,0.35,0.75} % javadoc

\lstset{basicstyle=\ttfamily}

%\geometry{a4paper,left=25mm,right=50mm, top=13mm, bottom=20mm}
%\geometry{a4paper,right=50mm,left=25mm}
\newcommand{\ff}{\triangleright}
\newcommand{\pref}[1]{\mathit{pref{#1}}}
\renewcommand{\min}{\mathit{Min}\;}

\newtheorem{def1}{Definition}
\newtheorem{satz}{Satz}
\newtheorem{lem}{Lemma}
\newtheorem{bew}{Beweis} 
\newtheorem{gleich}{Gleichung}[section]
\newtheorem{eigen}{Eigenschaft}
\newtheorem{folg}{Folgerung}
\theoremstyle{remark}
\newtheorem{beispiel}{Beispiel}

\author{Robert Hartmann}
\title{Vergleich von SQL-Anfragen: Theorie und Implementierung in Java}
\date{27. August 2013}
\parindent 0pt
\begin{document}

\pagestyle{empty}

\begin{center}

{\Large\sc Masterarbeit}

\vspace{1.25cm}

{\fontsize{22}{22}\selectfont Vergleich von SQL-Anfragen\\Theorie und Implementierung in Java}

\vspace{1.25cm}

{\Large\sc

Robert Hartmann

\vspace{.15cm}

Betreuer: Prof. Dr. Stefan Brass

\vspace{.15cm}

24. September 2013

}

\vspace{10.5cm}

\begin{tabular}{c}
	\includegraphics[height=15ex]{Bilder/siegel}
\end{tabular}

\vspace{0.5cm}

\rule{.7\textwidth}{.40pt}

\vspace{.2cm}

{\large\sc

martin-luther-universit"at halle-wittenberg

}

\end{center}

\cleardoublepage


%\include{dedication}



\onehalfspacing
\vspace*{10mm}
{\huge \textbf{Zusammenfassung}}\\\\
In der Lehre möchte man häufig SQL-Anfragen des Lernenden mit einer hinterlegten Musterlösung vergleichen, um deren Äquivalenz zu zeigen. Diese Arbeit beschäftigt sich mit dem semantischen Vergleich zweier SQL-Anfragen. Dabei wird ein Standardisierungsverfahren für SQL-Anfragen entwickelt, welches mit Hilfe des Datenbankschemas SQL-Anfragen legal umformt und dabei erreicht, dass semantisch gleiche Anfragen nach der Umformung auch syntaktisch gleich sind. Schlägt dieses Verfahren fehl, so wird in einem weiteren Verfahren mit realen Daten geprüft, ob die Anfragen tatsächlich ungleich sind. Diese Verfahren werden in einem Prototypen in Java implementiert und bilden eine Lernplattform, die dem Nutzer per Java Server Pages in einem Webbrowser zur Verfügung gestellt wird. Metadaten und Struktur der beiden SQL-Anfragen werden ausgewertet und der Nutzer erhält dadurch detailliertes Feedback, um so seine Fehler leichter zu bemerken.\newline\newline\newline\newline\newline
{\huge \textbf{Abstract}}\\\\
In university teaching one often want to compare a students sql query to a sample solution to show the equivalence of them. This thesis is about the semantic comparison of two SQL queries. We develop a method of standardization that legally rewrites a SQL query with the help of the database schema. If two SQL queries are semantically equal they shall be syntactically equal as well after the standardization. If this method fails we try to prove that the two queries are not semantically equal. Therefore we execute both queries on a real database and check their results. These methods are implemented in a Java prototype and form a learning platform that contains an interface built with Java Server Pages so the user can access it via web browser. The structure and meta information of the user query and the sample solution query are being compared and evaluated to give the user a detailed feedback on his query.

\pagestyle{plain}

\singlespacing
\tableofcontents

\onehalfspacing

\chapter{Einleitung / Motivation}
\label{chap:introduction}
%einleitung
In dieser Arbeit untersuchen wir die Operation Fortsetzung f"ur formale Sprachen, welche in der Arbeit [St87] eingef"uhrt wurde. Dabei bezeichnen wir die Fortsetzung eines Wortes $w$ in eine Sprache $L$, als das Minimum aller W"orter aus $L$, in denen $w$ ein Pr"afix ist. 
\\Man stelle sich einen Baum mit Wurzel $w$ vor und jeder Elternknoten $w'$ hat Kindknoten $w'x$ f"ur alle $x\in X^*$.
Wir folgen nun allen Pfaden von der Wurzel nach $X^*$. Treffen wir auf einem Pfad auf ein Wort aus $L$, so f"ugen wir dieses Wort dem Ergebnis hinzu und gehen diesen Pfad nicht mehr weiter.\\
Wir definieren die Fortsetzung einer Sprache $L$ in eine Sprache $W$ als Vereinigung der Fortsetzungen aller W"orter aus $L$ in $W$.\\\\
Wie in der Arbeit [St87] behandelt, erleichtert die Operation Fortsetzung die Durchschnittsbildung beim $\delta$-Limes zweier Sprachen. Mehr dazu im zweiten Abschnitt dieser Arbeit.\\\\
Zuerst legen wir die verwendete Notation fest. Dann betrachten wir die algebraische Struktur $({2^X}^*,\ff)$ und untersuchen diese auf typische Eigenschaften.
Da diese Struktur nicht kommutativ ist, untersuchen wir dann die Stabilit"at der Operation Fortsetzung in Bezug auf die mengentheoretischen Operationen $\cap,\cup$ und die Konkatenation im Vorder- sowie Hinterglied.
Au"serdem wird in diesem Abschnitt die Monotonie der Operation Fortsetzung bez"uglich $\subseteq$ betrachtet.
\\Im dann folgenden Abschnitt untersuchen wir die Operation Fortsetzung, wenn eine der beiden Operanden die spezielle Form $\pref(L)$ oder $W\cdot X^*$ hat. Abschlie"send betrachten wir die Abgeschlossenheitseigenschaften der Operation Fortsetzung f"ur die Klassen der CHOMSKY-Hierachie.





\chapter{Forschungsstand und Einordnung}
\label{chap:forschung}
\section{Einleitung}

Die Idee, SQL-Anfragen von Lernenden automatisch zu kontrollieren, ist nicht völlig neu. Weil eine Auswertung über den Standard SQL-Parser nicht sehr umfangreich ist und dieser semantischen Fehlern kein sinnvolles Feedback gibt, sind bereits einige Ansätze veröffentlicht worden, die es sich zum Ziel gemacht haben eine SQL-Anfrage näher zu analysieren. Verschiedene Projekte beschäftigen sich dabei \mbox{z. B.} mit dem Aufdecken von semantischen Fehlern. Andere Plattformen konzentrieren sich auf den Lernerfolg, den der Student erreichen soll, und analysieren die Art der Fehler des Lernenden. Damit erreicht man eine Zuteilung von passenderen Aufgaben, sodass der Lernende weder gelangweilt noch überfordert ist. %matchamking

In diesem Abschnitt möchten wir die bereits existierenden Ansätze auf diesem Gebiet betrachten, um dann diese Arbeit einordnen zu können.

\section{SQL-Tutor}

In \cite{sqltut1} beschreibt Antonija Mitrovic ein Lernsystem, das SQL-Tutor genannt wird. Nach Auswahl einer Schwierigkeitsstufe wird dem Studenten ein Datenbankschema und eine Textaufgabe vorgelegt. Der Student hat nun ein Webformular vor sich, in dem sich für jeden Teil der SQL-Anfrage ein Eingabefeld befindet. So werden die Anteile \verb|SELECT|, \verb|FROM|, \verb|WHERE|, \verb|ORDER BY|, \verb|GROUP BY| sowie \verb|HAVING| einzeln eingetragen.

Der SQL-Tutor analysiert nun die Anfrage des Studenten und gibt spezifisches Feedback. Dabei wird nicht nur geklärt, ob die Anfrage korrekt ist, sondern auch, was bei einer falschen Eingabe genau fehlerhaft ist. Das reicht von konkreten Hinweisen auf den spezifischen Teil der Anfrage bis hin zu eindeutigen Hinweisen wie >>Musterlösung enthält einen numerischen Vergleich mit der Spalte a, ihre Lösung enthält aber keinen solchen Vergleich<<.

Umgesetzt wird dieses Programm durch 199 fest einprogrammierte Constraints. Dadurch ist es potentiell möglich bis zu 199 spezifische Hinweismeldungen für den Studenten bereitzustellen. Das reicht von syntaktischen Analysen wie >>Der SELECT-Teil von einer Lösung darf nicht leer sein.<< bis hin zu semantischen Analysen, gepaart mit Wissen über die Domain (Datenbankschema und Musterlösung), bei denen die Lösung des Studenten mit der Musterlösung und dem Datenbankschema verglichen wird. Insbesondere versucht der SQL-Tutor Konstrukte wie numerische Vergleiche mit gewissen Operatoren in der Lösung des Studenten zu finden, wenn diese in der Musterlösung auftauchen. Auch komplexere Constraints, die sicherstellen, dass bei einem numerischen Vergleich \verb|a > 1| das gleiche ist wie \verb| a >= 0| sind vorhanden. 

Allerdings gibt es auch hier Schwächen. Da der verwendete Algorithmus die Constraints nach einander abarbeitet, kann es zu unnötigen Analysen der Anfrage kommen und damit auch zu einem unnötigen Zeitaufwand. Nach eigenen Tests werden manche äquivalente Bedingungen nicht erkannt. So wird \verb|a < 0| für richtig, aber \verb|0 > a| für falsch gehalten. Ähnlich verhält es sich, falls eine der Argumente des Vergleichs das Ergebnis einer Unterabfrage ist. Die Constraints sind fest einprogrammiert und nicht von der Anfrage abhängig und damit genügt es für eine neue Aufgabe Text und Musterlösung einzulesen. 

Der SQL-Tutor lässt außerdem den eingesendeten Lösungsvorschlag auf einer SQL-Datenbank mit Testdaten laufen und vergleicht die Tupel mit den Antworttupeln, die man mit der gespeicherten Musterlösung erhält.

\subsection*{Abgrenzung zum SQL-Tutor}

Der Grundgedanke des SQL-Tutors überschneidet sich durchaus mit dem Ansatz dieser Arbeit. Ein Grundpfeiler des SQL-Tutors ist es, dem Studenten detailliertes Feedback über seine semantischen und syntaktischen Fehler zu geben. Das Programm was im Zuge dieser Arbeit entsteht soll weniger semantische Fehler analysieren, als viel mehr versuchen zwei SQL-Anfragen zu vergleichen, unabhängig davon wie sie aufgeschrieben sind. Des Weiteren bedient sich der SQL-Tutor einer Testdatenbank mit realen Testdaten. Unser Programm soll nur das Datenbankschema kennen und ohne Daten bestimmen, ob zwei Anfragen das gleiche Ergebnis liefern. Erst in einem optionalen zweiten Schritt prüfen wir in unserem Programm, ob beide Anfragen gleiche Ergebnisse auf konkreten Daten liefern. Hat bereits der erster Schritt ein positives Ergebnis gemeldet, dann ist der zweite Schritt unnötig. Der SQL-Tutor operiert allerdings zur Ermittlung der Übereinstimmung nur auf Testdaten. Bei ungünstig gewählten Testdaten kann es passieren, dass der Eindruck entsteht zwei Anfragen wären gleich, obwohl sie es nicht sind. Entstehen kann dies, weil auf den vorhanden Testdaten zufällig beide Anfragen die gleichen Ergebnisse liefern könnten.

\section{SQL-Exploratorium}

Im Artikel \cite{explora1} werden SQL-Lernplattformen in zwei Kategorien eingeteilt. Die erste Kategorie zeichnet sich durch existieren Plattformen aus, welche durch Multimedia versuchen dem Lernenden einzelne Bestandteile der Sprache SQL bildlich darzustellen. Hierfür werden meist Websites mit Multimediainhalten erstellt. Die zweite Kategorie beinhaltet Software, welche die Lösung eines Lernenden analysiert und konkrete Hinweismeldungen gibt. Dazu zählt auch der eben beschriebene SQL-Tutor.

Das SQL-Exploratorium macht es sich nun zur Aufgabe die beiden Ansätze zu verbinden und stellt sich dabei hauptsächlich verwaltungstechnische Fragen wie \mbox{z. B.} 

\begin{itemize}
\item Wie ermögliche ich dem Studenten Zugriff auf verschiedene Lernsysteme, ohne sich mehrfach einloggen zu müssen?
\item Wie können Lernerfolge in einem System einem anderen nutzbar gemacht werden?
\item Wie kann man aus mehreren Logfiles der eingereichten Lösungen eines Studenten, von unterschiedlichen Systemen, einen Wissensstand des Studenten ableiten?
\end{itemize}

Da diese Fragen wenig Relevanz für diese Arbeit haben, betrachten wir im Folgenden welche einzelnen Plattformen für das SQL-Exploratorium genutzt werden.

\subsection{Interactive Examples}

Über eine Schnittstelle, die sich WebEX nennt, hat der Student Zugriff auf insgesamt 64 Beispielanfragen. Wählt man eine Anfrage aus können Teile davon in einer Detailansicht geöffnet werden. Dem Studenten wird dann ausführlich erklärt, was die einzelnen Teile der Anfrage genau bewirken. Sowohl die Beispielanfragen, als auch die Hinweise sind manuell erzeugt und abgespeichert. Hier wird nichts automatisch generiert, daher ist dieses Projekt nicht relevant für die Arbeit. Der Lernerfolg des Studenten wird hier über die ein >>click-log<< geführt, das bedeutet es wird aufgezeichnet, was der Student wann und in welcher Reihenfolge angeklickt hat. So ist es \mbox{z. B.} möglich herauszufinden welche Teile einer bestimmten Anfrage besonders interessant für den Lernenden sind.

\subsection{SQL Knowledge Tester}

Der SQL Knowledge Tester, im Nachfolgendem SQL-KnoT genannt, konzentriert sich darauf Anfragen eines Studenten zu analysieren. Dabei wird dem Studenten zur Laufzeit eine Frage generiert. Dabei werden vorhandene Datenbankschemata in einer bestimmten Art und Weise verknüpft und Testdaten sowie eine Frage für den Studenten generiert. Dies geschieht mit fest einprogrammierten 50 Templates, die in der Lage sind über 400 Fragen zu erzeugen. Zu jeder Frage werden zur Laufzeit Testdaten für die relevanten Datenbanken erzeugt. Ausgewertet wird die Anfrage des Studenten dann, indem die zurückgelieferten Tupel der Studentenanfrage verglichen werden mit den Tupeln, welche die Musterlösung erzeugt. 

\subsection*{Abgrenzung zur Arbeit}

Erwähnenswert ist, dass initial keine Daten existieren. Wie beim Ansatz dieser Arbeit existieren zunächst nur Datenbankschemata. Die Daten und auch die Aufgabe an den Studenten werden aus Templates generiert. Die Auswertung erfolgt dann allerdings durch den Vergleich der zurückgelieferten Tupel der Muster- und Studentenanfrage. Hierbei kann wieder das Problem auftreten, dass für beide Anfragen die erzeugten Testdaten die gleichen Tupel zurückliefern, es aber bei einem anderen Zustand sein kann, dass sich die Tupelmengen unterscheiden. 

Der Ansatz vom SQL-KnoT ist durchaus interessant, wird aber in dieser Arbeit nicht weiter ausgeführt, da wir in dieser Arbeit keine Testdaten erzeugen möchten. Wir benutzen vielmehr Beispieldaten als optionalen zweiten Schritt.

\subsection{Weiteres}

\subsubsection{Adaptive Navigation for SQL Questions}

Hierbei handelt es sich nur um ein Tool, was aufgrund früherer Antworten des Studenten, diesem möglichst passende neue Fragen vorlegen soll. Dieser Teil des SQL-Exploratoriums dient also dazu, den Wissensstand des Studenten festzustellen und ist für diese Arbeit daher unerheblich. 

\section{WIN-RBDI}

Das Programm WINRBDI, welches in \cite{winrbdi1} beschrieben wird verfolgt einen weiteren, interessanten Ansatz. Anstelle von fest vorgegebenen Demoanfragen, wird die eingegebene Anfrage zunächst in esql eingebettet. Die Ausführung der Anfrage wird dann schrittweise durchgeführt. Der Student hat also die Möglichkeit die Anfrage im Schrittmodus, ähnlich eines Debugger, oder im Fortsetzen-Modus auszuführen. Im Schrittmodus wird jeder Teilschritt der Abarbeitung der Anfrage aufgezeigt. Es werden temporär erzeugte Tabellen angegeben sowie auch eine Erklärung welcher Teil der Anfrage für den aktuellen Abarbeitungsschritt verantwortlich ist. So soll es dem Studenten möglich sein, die unmittelbaren Konsequenzen seiner SQL-Anfrage für die Abarbeitung zu begreifen. 

Des Weiteren hilft dieser Ansatz dem Studenten die Abarbeitung einer Anfrage zu Visualisieren, indem die von der WHERE-Klausel betroffene Spalten markiert werden. Dies hilft gerade Lernanfängern bei der Visualisierung von Konzepten wie JOINs.

\subsection*{Abgrenzung zur Arbeit}

Dieser Ansatz hebt sich von den bisherig betrachteten Ansätzen ab. Hier wird dem Studenten durch eine Visualisierung der Ausführung der Anfrage versucht deutlich zu machen, welche Teile der formulierten Anfrage was genau bewirken. Für den Lernerfolg des Studenten ist dies sicherlich hilfreich, zumal eine Visualisierung stets hilft, Zusammenhänge leichter zu begreifen. Diese Arbeit verfolgt allerdings ein anderes Ziel, da sie zwei SQL-Anfragen miteinander vergleicht und nicht versucht die Abarbeitung einer Anfrage verständlicher zu machen.

\section{SQLLint}

>>SQLLint<<, ein Semantik-Prüfer für SQL-Anfragen, beschäftigt sich mit semantischen Fehlern in SQL-Anweisungen, welche unabhängig vom Datenbankzustand auftreten. Dabei behandelt das Projekt Anfragen, von denen man ohne Kenntnis der Aufgabenstellung sagen kann, dass sie, in der vorliegenden Form, nicht beabsichtigt sind. Dies ist wahrscheinlich, wenn man \mbox{z. B.} Teile aus der Anfrage herausstreichen kann ohne die Funktion der Anfrage zu verändern.  Das Problem besteht darin, dass aktuelle DBMS-Systeme solche Anweisungen ohne Fehler- oder Warnmeldung ausführen. Der Nutzer, also insbesondere der lernende Nutzer, ist somit kaum in der Lage überhaupt zu bemerken, dass es einen Fehler in seiner Anfrage gab. Die allgemeine Frage, ob eine Anfrage semantische Fehler enthält ist nicht entscheidbar. Dennoch macht es sich SQLLint zur Aufgabe eine große, typische Teilmenge von SQL-Anfragen zu bearbeiten. Ziel des Projektes ist es, mit semantischen Warn- und Fehlermeldungen, die Codeentwicklung zu beschleunigen und die Anzahl der Fehler darin zu verringern.

Eine wichtige Zielstellung des Projektes ist es, solche Fehlermeldungen in der Lehre einzusetzen. In \cite{sqllint1} wird auch deutlich gemacht, das eine Motivation dieses Projektes aus typischen Fehlern von Studenten entspringt. So wurde im selben Artikel aufgeführt, dass semantische Fehler bei Lernenden am häufigsten auftreten. Unter den drei häufigsten semantischen Fehlern befinden sich: fehlende JOIN Bedingung, (zu) viele Duplikate, unnötiger JOIN. Diese Fehler machen bereits ca. 37 Prozent der semantischen Fehler aus.

Weiterhin fällt auf, dass die Anzahl syntaktischer Fehler mit fortschreitendem Schwierigkeitsgrad der SQL-Anfrage steigen, aber die Anzahl semantischer Fehler nahezu unabhängig von jenem Schwierigkeitsgrad ist. Einfache Anfragen haben sogar zwei mal mehr semantische Fehler als syntaktische Fehler. Siehe dazu Abbildung 4 in \cite{sqllint1}.

\subsection{Algorithmus zum Finden von inkonsistenten Bedingungen}

Die Algorithmen im SQLLint-Projekt sollen inkonsistente Bedingungen finden. Dieses Problem ist allerdings im Allgemeinen unentscheidbar. Dennoch ist es möglich Teilmengen von Anfragen anzugeben, für die man die Konsistenz algorithmisch Entscheiden kann. Folgende Ausführungen zum Algorithmus entstammen der Arbeit >>Proving the Safety of SQL Queries<< von Stefan Brass und Christian Goldberg \cite{brass1}.

Konsistenz im diesem Sinne soll bedeuten, dass es ein endliches Modell (relationaler Datenbankzustand, manchmal auch Datenbankinstanz genannt) existiert, sodass das Ergebnis der Anfrage nicht leer ist.

Wir nehmen im Folgenden an, dass die SQL-Anfragen keine Datentyp Operationen enthalten. Alle atomaren Formeln haben also die Form $t_1\theta t_2$ mit $\theta\in \{=,<>,<,<=,>,>=\}$ und $t_1,t_2$ sind Attribute oder Konstanten. Aggregationsfunktionen sind noch Bestandteil der Forschung und werden daher nicht behandelt.

\subsection{Bedingungen ohne Unteranfragen}

WHERE-Bedingungen, die keine Unteranfrage enthalten, können mit bestimmten Methoden entschieden werden. Ein Beispiel dafür sind die Algorithmen von Guo, Sun und Weiss \cite{decideable1}. Als erster Schritt wird die Negation \verb|NOT| so weit wie möglich an die atomaren Formeln weitergereicht, indem die \verb|DE-MORGAN|'sche Regeln angewendet werden. Dadurch drehen sich die Vergleichsoperatoren um, wir sprechen hierbei von dem >>gegenteiligem Operator<<. Die Menge $O=\{ \{\leq,>\} , \{\geq,<\} , \{ =, =\}, \{\neq,\neq\} \}$ enthält jeweils 2er Mengen von einem Operator und seinem >>gegenteiligem Operator<<. Im nächsten Schritt wird die Bedingung in die disjunktive Normalform (DNF) umgeformt, so dass folgende Struktur entsteht: $\phi_1 \vee ... \vee \phi_n$. Diese ist genau dann konsistent, wenn mindestens ein $\phi_i$ konsistent ist. Nun können wir die Methoden aus \cite{decideable1} anwenden. Im wesentlichen handelt es sich dabei um einen gerichteten Graphen, in dem Knoten markiert sind mit >>Tupelvariable.Attribut<< und Kanten mit $<$ oder $\leq$. Dann werden Intervalle von möglichen Werten für jeden Knoten berechnet. Dabei ist zu beachten, dass die SQL-Datentypen, wie \verb|NUMERIC(1)|, das Intervall zusätzlich einschränken.
Wenn es endlich viele mögliche Werte für einen Knoten gibt, dann können Ungleich-Bedingungen ($t_1<>t_2$) zwischen Knoten wichtig werden und ein Graphfärbungsproblem kodieren. Daher erwarten wir keinen effizienten Algorithmus, wenn es viele $<>$ Bedingungen gibt. In allen anderen Fällen ist die Methode in \cite{decideable1} schnell. Anzumerken ist allerdings noch, dass die Umwandlung in DNF zu exponentiellem Wachstum in der Größe der Anfrage führen kann.

\subsection{Unteranfragen}

Um unnötige Betrachtungen zu vermeiden, beschäftigt sich das SQLLint-Projekt nur mit \verb|EXISTS|-Unteranfragen. Alle anderen Unteranfragen (\verb|IN,>=ALL,| etc.) können auf die \verb|EXISTS|-Unteranfrage reduziert werden. Oracle führt solche Umwandlungen durch, bevor der Optimierer beginnt an der Anfrage zu arbeiten.

Die Idee zur Behandlung von Unteranfragen stammt aus bekannten Methoden der automatischen Beweiser. Hierzu wird in der Arbeit \cite{brass1} eine Variante der Skolemisierung vorgestellt. Das genaue Vorgehen wird in in jenem Artikel erklärt.

\subsection{Unnötige logische Komplikationen}

Es kann vorkommen, dass eine Teilbedingung inkonsistent ist, die gesamte Bedingung allerdings dennoch konsistent ist (Aufgrund der Disjunktion). Ebenso denkbar ist der umgekehrte Fall, dass also Unterbedingungen Tautologien sind. Beide Vorkommnisse sind vermutlich nicht gewollt und können zu einem unerwünschte Verhalten einer Anfrage führen. Wie in \cite{brass2} festgestellt wurde, werden in Klausuren von Studenten auch öfter unnötige Bedingungen angegeben, welche bereits per Definition impliziert werden. Als Beispiel betrachten wir die Bedingung \verb|A IS NOT NULL|. Diese wird unnötig, wenn wir wissen, dass \verb|A| bereits als \verb|NOT NULL| definiert ist.

Im Folgenden wird in \cite{brass2} eine mögliche Formalisierung der Voraussetzung für >>keine unnötigen logischen Komplikationen<< erläutert. Immer wenn in der DNF der Anfragebedingung eine Unterbedingung mit >>true<< oder >>false<< ersetzt wird, ist das Ergebnis nicht zur Ausgangsbedingung äquivalent.

Realisiert wird dies durch eine Reihe von Konsistenzprüfungen. Es sei die DNF der Anfragebedingung $C_1\vee ...\vee C_m$, mit $C_i=(A_{i,1}\wedge ...\wedge A_{i,n_i})$. Unser Kriterium ist genau dann erfüllt, wenn die folgenden Formeln alle konsistent sind:

\begin{enumerate}
\item $\neg(C_1 \vee ... \vee C_m)$ - Die Negation der gesamten Formel. Ansonsten könnte man diese durch >>true<< ersetzen.
\item $C_1 \wedge \neg(C_1 \vee ... \vee C_{i-1} \vee C_{i+1} \vee ... \vee C_m)$ mit $i\in [1,m]\cap \mathbb{N}$. Ansonsten könnte $C_i$ mit >>false<< ersetzt werden.
\item  $A_{i,1} \wedge ... \wedge A_{i,j-1} \wedge  \neg A_{i,j} \wedge A_{i,j+1} \wedge ... \wedge A_{i,n_i} \wedge \neg(C_1 \vee ... \vee C_{i-1} \vee C_{i+1} \vee ... \vee C_m)$ mit $i\in [1,m] \cap \mathbb{N}$, $j\in [1,n_i]\cap \mathbb{N}$. Ansonsten könnte $A_{i,j}$ mit >>true<< ersetzt werden.
\end{enumerate}

Zu weiteren unnötigen logischen Komplikationen zählen zu allgemeine Vergleichsoperatoren ($>=$ anstelle von $=$). Weiterhin gehören unnötige JOINs zu einem wichtigen Typ von unnötigen logischen Komplikationen.

\subsection{Laufzeitfehler}

Als Bemerkung ist festzuhalten, dass sich das SQLLint-Projekt auch mit Laufzeitfehlern beschäftigt. Als Beispiel stelle man sich folgende SQL-Bedingung vor: \verb|A=(SELECT ...)| Es muss hier sichergestellt werden, dass die \verb|SELECT|-Unteranfrage nur einen Rückgabewert hat. Solche Fehler sind schwierig zu finden, da sie nicht immer Auftreten müssen. 

Wie das SQLLint-Projekt damit umgeht, soll hier nicht weiter besprochen werden. Details dazu sind zu finden in \cite{brass2}. 

\subsection*{Zusammenhang zu dieser Arbeit}

Obwohl SQLLint auf den ersten Blick eine andere Zielstellung als diese Arbeit verfolgt, so sind doch einige Ansätze deckungsgleich. Einige der Ansätze von SQLLint können Grundlagen für diese Arbeit sein. Der Ansatz der Standardisierung der SQL-Anfragen ist mit Umwandeln der Formeln in eine Normalform ein guter Ansatzpunkt und zeigt wie sich komplexe Bedingungen vereinheitlichen lassen. Auch die Erkenntnis, dass sich alle Unteranfragen auf \verb|EXISTS| Unteranfragen reduzieren lassen, wird helfen die Unteranfragen zu standardisieren. Dadurch wird die Vielfalt der Unteranfragen eingeschränkt und ein Vergleich zweier SQL-Anfragen wird vereinfacht.

Weiterhin könnte man in späteren Ausbaustadien des Programmes, welches im Rahmen dieser Arbeit entsteht, die Funktionalitäten des SQLLint einbeziehen . Dies würde die Art des Feedbacks für den Lernenden deutlich verbessern, da wir uns in dieser Arbeit zunächst auf das Vergleichen von zwei SQL-Anfragen konzentrieren. Dabei stehen vor allem Hinweise im Vordergrund, die dem Lernenden zeigen sollen, warum seine Lösung mit der Musterlösung noch nicht übereinstimmen kann.

Ein davon unabhängiges Feedback für die Anfrage des Lernenden würde den Lernverlauf stark beschleunigen und mit hoher Wahrscheinlichkeit sogar die Fehler der Anfrage eliminieren, so dass die Anfrage dann auf die Musterlösung passt.

\chapter{Verwendete Software}
\label{chap:software}
\section{SQL Parser}

\subsection{über den SQL Parser: ZQL}

\subsection{Parserbäume}

\subsection{Grenzen des Parsers}

\section{Java Server Pages}

\subsection{Überblick}

\subsection{Einbettung in JSP}

\subsection{Log}

\chapter{Theoretische Betrachtungen}
\label{chap:theorie}
Um die Frage zu beantworten wie man zwei SQL Anfragen miteinander vergleichen kann, muss man sich zunächst die Struktur einer solchen Anfrage betrachten. Exemplarisch betrachten wir im folgenden \verb|SELECT| Anfragen. Es werden mehrere Ansätze in diesem Teil der Arbeit verfolgt, wie man die Gleichheit von zwei Anfragen zeigen kann. Offensichtlich sind zwei SQL-Anfragen semantisch äquivalent, wenn sie ebenfalls syntaktisch korrekt sind. Interessanter sind daher Anfragen, die zunächst nicht syntaktisch dekungsgleich sind. 

Ein Ansatz besteht darin beide SQL-Anfragen einer Standardisierung zu unterziehen. Wie genau so etwas durchgeführt werden kann, wird im Folgenden noch erläutert. Wir würden dann zwei standardisierte SQL-Anfragen erhalten. Sind diese syntaktisch äquivalent, so handelt es sich um identische Anfragen. Dieser Ansatz wird uns mit einigen Problemen konfrontieren und daraus entwickeln wir einen zweiten Ansatz. 

Dieser versucht durch gleichartige Umformungen, die zwei Anfragen zu unifizieren (gleich zu machen). Bei diesem Ansatz würden wir also versuchen die geparsten Operatorbäume miteinander zu vergleichen. Auch diese Lösung birgt Vorteile aber auch Probleme mit sich, die im Folgenden besprochen werden.

\section{Hintergrund}

Es gibt syntaktisch unterschiedliche Anfragen, die jedoch semantisch äquivalent sind. So liefern die folgenden Anfragen die gleichen Ergebnisse, sind aber nicht syntaktisch äquivalent.

\begin{verbatim}
SELECT * FROM emp e WHERE e.enr > 5
\end{verbatim}

\begin{verbatim}
SELECT * FROM emp e WHERE 5 < e.enr
\end{verbatim}

\begin{verbatim}
SELECT * FROM emp e WHERE e.enr >= 6
\end{verbatim}

Wie man leicht sieht, sind die Anfragen ähnlich. Im folgenden werden zwei Strategien besprochen, welche beide zum Ziel haben, zwei SQL-Anfragen miteinander zu vergleichen.

Neben solchen syntaktischen Varianten, kann es auch sein, dass unnötige Bedingungen aufgeschrieben werden, die das Ergebnis nur unnötig kompliziert machen. Eine Möglichkeit ist folgende Anfrage, in der offensichtlich die letzte Bedingung überflüssig ist.
\begin{verbatim}
SELECT * FROM emp e WHERE e.enr > 5 AND e.enr <> 2
\end{verbatim}

Unser Programm müsste nun erkennen, dass das Attribut \verb|enr| bereits beschränkt ist, und der Wert 2 gar nicht mehr auftreten kann. Das Programm, was zu dieser Arbeit entwickelt wird, kann mit solchen redundanten Eigenschaften nicht umgehen. Es wäre auch mehr ein Problem für einen ``semantic checker'', da es hier gar nicht auf zwei verschiedene Anfragen ankommt. Hier ist bereits diese eine Anfrage in sich selbst zu kompliziert. Mit derlei Problemen beschäftigt sich das Projekt ``SQLLint'' der Martin-Luther-Universität Halle-Wittenberg, mehr dazu im Artikel \cite{brass1} und \cite{brass2}.

Weitere Probleme sind Operatoren, die sich auf andere abbilden lassen. Man kann dann nie wissen, in welcher Art und Weise der Lernende die Aufgabe formulieren wird. Man betrachte sich dazu folgende zwei Anfragen:
\begin{verbatim}
SELECT * FROM emp e WHERE e.sal BETWEEN 10 AND 200

SELECT * FROM emp e WHERE e.sal >= 10 AND e.sal <=200
\end{verbatim}

Offensichtlich sind die Anfragen äquivalent. Dies erreichen wir im wesentlichen, in dem wir bestimmte Operatoren wie \verb|BETWEEN| abschaffen und durch die äquivalenten Ungleichungen mit \verb|>=| und \verb|<=| ersetzen. Ähnliches gibt es für \verb|INNER JOIN| im \verb|FROM| Teil, mit Ersetzung durch Vergleiche im \verb|WHERE| Teil. 

\section{Workflow}

\begin{verbatim}
INPUT: QUERY Q1,Q2;
P1 = preprocessing(Q1);
P2 = preprocessing(Q2);
compare(P1,P2); // possible warnings can be displayed now
ANSWER = match(Q1,Q2);
if ANSWER yes then
    /* If that worked, we know both solutions are the same */
    display success
else 
    if do_real_db_compare(Q1,Q2) then
        /* now we don't know if they are the same because
         * they couldn't be matched but test on real data 
         * showed the correct results 
         display may be correct
    else 
        /* if the real data test failed we have a proof 
         * in form of a data set, that both querys can't be the same */
         display fail
    endif
endif
output result of compare(P1,P2)
/* The result may show the cause of a fail or a ``may be'' solution. 
 * It can provide hints so that the student can improve.
 * Even if the solution was correct i.e. it was matched with the sample solution, 
 * it may be that the students soltion contained unnecassary joins, or formulas. */
\end{verbatim}

\section{Preprocessing}

Im Abschitt >>Forschungsstand<< haben wir bereits einige Lernplattformen/projekte zum Thema SQL kennen gelernt. Viele dieser Plattformen möchten dem Lernenden genügend Feedback beim Lernprozess geben. Dies ist nicht nur sinnvoll, damit der Student schneller auf korrekte Lösungen stößt, sonder auch, weil die Standardhinweise eines SQL Systems meist nur auf syntaktische Fehler hinweisen. Einen großen Beitrag zur Verbesserung von Fehlermeldungen hat das Projekt SQLLint vorzuweisen, da es Fehlermeldungen und Hinweise konkreterer Natur ausgibt. Hervorzuheben ist, noch einmal, dass es sich hierbei um semantische Fehlermeldungen handelt. Schon nach kurzer Einlernzeit sinken die Anzahl an syntaktischen Fehlern bei Lernenden. Dafür machen diese mehr semantische Fehler, was um so schlimmer ist, da bisher kaum oder keine Warnhinweise für solche Fehler existierten. 

Dennoch sollen in dieser Arbeit zwei SQL-Anfragen verglichen werden. Wir können hier also nicht alle Ideen des SQLLint übernehmen. Egal ob das Matchen der Musterlösung und der Lösung des Lernenden gelingt oder nicht, wir möchten dem Lernenden Feedback geben, an dem er möglicherweise sehen kann, warum das Matching nicht gelungen ist. Wir können dabei, wie bereits erwähnt, nicht an die Komplexität des SQLLint anknüpfen. Stattdessen werden wir uns eines einfachen Sammelns von Metainformationen der SQL-Anfrage bedienen. Diese sammeln wir bevor die zwei Anfragen durch Folgeschritte angepasst oder verändert werden. Am Ende des Matchingsversuchs sollen Metainformationen der zwei Anfragen verglichen werden und dem Lernenden soll Feedback gegeben werden. Konnte keine Übereinstimmung der zwei Anfragen erreicht werden, so können die Metainformationen dem Lernenden Anhaltspunkte für eine richtige Lösung geben. Konnten die Anfragen unifiziert werden, so sind die Metainformationen dennoch von Interesse. Es könnte sein, dass der Lernende eine unnötig komplexe Lösung eingesandt hat, die sich durch Anpassungen vereinfachen ließe. So kann der Lernende potentiell auch aus einer korrekten Lösung noch etwas lernen.

Wir möchten für jede SQL-Anfrage ein Preprocessing vor der eigentlichen Bearbeitung vorschalten, was im wesentlichen folgende Punkte beinhalten soll.

\begin{itemize}
\item Anzahl der JOIN Bedingungen
	\begin{itemize}
	\item Anzahl von OUTER/INNER Joins
	\end{itemize}
\item Anzahl atomarer Formeln in \verb|WHERE|-Teil
\item Anzahl atomarer Formeln in \verb|HAVING|-Teil
\item Anzahl Tabellen in \verb|FROM|-Teil
\item Anzahl Attribute im \verb|SELECT|-Teil *
\item existiert ein \verb|DISTINCT|
\item existiert ein \verb|GROUP BY| *
	\begin{itemize}
	\item wenn ja, stimmen die Attribute überein?
	\end{itemize}	 
\item existiert ein \verb|HAVING BY|
\item existiert ein \verb|ORDER BY| und ist es notwendig? (\verb|ORDER BY ... ASC|) *
\item Tiefe des Parserbaums, kann Aufschluss über unnötige Klammerung geben. Siehe dazu Abschnitt >>Wie funktioniert der Parser<<
\end{itemize}

Unterscheiden sich Musterlösung und Lösung des Studenten in den mit * markierten Punkten ist es extrem unwahrscheinlich, dass beide Lösungen die gleichen Tupel zurückliefern würden. Hier möchten wir im Vorfeld dem Lerneden eine Warnung anzeigen, dass er höchstwahrscheinlich etwas vergessen hat. Alle anderen Punkte werden im Anschluss an die eigentliche Analyse der Anfragen abgeglichen. So sind etwa folgende Meldungen denkbar:

\begin{itemize}
\item ``the sample solution contains two joins but your solution does not contain any join.''
\item ``Your solution is correct but the sample solution contains two less atomar formulas (formula1, formula2).''
\item ``Your solution is correct but the sample solution does not contain DISTINCT. Reconsidder if it is really necessary.''
\end{itemize}

Zusammenfassend kann man Folgendes sagen: Das Preprocessing wird direkt nach dem Parsen einer SQL-Anfrage durchgeführt. Es sammelt Metainformationen über die Anfrage. Da wir zwei Anfragen vergleichen, werden diese Metainformationen einzeln für jede Anfrage gespeichert. Dann beginnen wir mit dem zweiten Schritt, dem Angleichen der SQL-Anfragen. Dazu verwenden wir Strategien, die in folgenden Kapiteln besprochen werden.

Egal ob die Ergebnisse im zweiten Schritt erfolgreich waren oder nicht, wir geben danach einen Vergleich der Metainformationen aus. Beispiele wurden eben bereits genannt. Das soll dem Lernenden bei falscchen Lösungen Anhaltspunkte geben, wie eine richtige Lösung aussehen könnte. Bei einer korrekten Lösung, können solche Hinweise trotzdem nützlich sein, denn die Anfrage des Lernenden kann ja trotz Korrektheit zu lang bzw. komnpliziert sein. Dies würde bei einem Vergleich der gesammelten Metainformationen deutlich werden.

\section{Standardisierung von SQL-Anfragen}

Zunächst verfolgen wir den Ansatz zwei SQL-Anfragen zu vergleichen, indem wir sie standardisieren. Die Kriterien der Standardisierung werden im Detail behandelt. Standardisiert man die Musterlösung, als auch die Lösung des Lernenden nach den gleichen Kriterien, so kann man danach durch einen einfachen Stringvergleich auf die Äquivalenz schließen. 

\subsection{Entfernen von syntaktischen Details}

Das Entfernen von syntaktischen Details übernimmt zum großen Teil bereits der Parser. Er entfernt unnötige Leerzeichen, Kommentare sowie unnötige Klammern. Aufgrund der Arbeitsweise des Parsers gibt es allerdings Situationen, in dem der Parser scheinbar nicht alle unnötigen Klammern entfernt. Wie im Abschnitt >>Verwendeter Parser<< erläutert wird, sind die geparsten Bäume nicht binär. Ein Baum wie in Abbildung \ref{baum1} zu sehen, ist daher zu vermeiden. 

Der Parser hilft allerdings dabei die SQL-Anfrage in einer Datenstruktur zu überführen, die frei von allen syntaktischen Details ist. Dazu gehören Leerzeichen, Tabs, Zeilenumbrüche und Groß/Kleinschreibung von Schlüsselwörtern.

\subsection{Vereinheitlichen der FROM Klausel}

Wir beginnen mit der Betrachtung der \verb|FROM| Klausel. Da die Reihenfolge der Spaltennamen im \verb|SELECT| Teil oft von der Aufgabenstellung vorgeschrieben ist, wird diese auch nicht verändert.

Im \verb|FROM| Teil werden zunächst alle auftretenden Tabellennamen lexikographisch sortiert. Danach werden automatische Aliase erzeugt. Sind bereits Aliase vergeben wurden, so werden diese ebenfalls durch die automatischen Aliase ersetzt. Eine Hashtabelle speichert frühere Zuweisungen, damit im \verb|SELECT| und \verb|WHERE| die Aliase ebenfalls korrekt ersetzt werden.

Hatten die vorkommenden Tabellen im \verb|FROM| Teil keinen Alias wird nur der küsntliche Alias eingeführt.

\begin{figure}
Eingabe: \\\verb|SELECT e.id, e.name, d.region FROM emp e, dep d WHERE e.depid = d.id|\\

Anpassung: \\\verb|SELECT a2.id, a2.name, a1.region FROM dep a1, emp a2 WHERE a2.depid = a1.id|\\
\caption{Beispiel: Umwandlung des FROM Teils einer SQL-Anfrage}
\end{figure}

\subsection{Umwandlung der WHERE Bedingung in KNF}

Aufgrund der Eigenheiten des ZQL-Parsers ist es möglich, dass eine unnötige Klammerung nicht entfernt wird. Beispiele dafür sind im Abschnitt >>ZQL-Parser<< zu finden. Es ist daher wünschenswert eine Normalform des \verb|WHERE| Teils zu erreichen. In diesem Fall wurde die konjunktive Normalform (KNF) gewählt.


\subsubsection{Entfernung unnötiger Klammerungen}

Ein Ausdruck \verb|((a > 5)  and ((b > 5) and (c > 5)))| enthält unnötige Klammern, da der Operator \verb|and| als Operand von einem weiteren \verb|and| vorkommt. Folgender Ausdruck ist äquivalent: \verb|((a > 5)  and (b > 5) and (c > 5))|. Diese spezielle Form der Klammerung entsteht aus der Tatsache, dass der ZQL-Parserbaum nicht binär ist und beide,  eben genannten, Beispiele nicht den gleichen Baum beschreiben. Als ersten Schritt in Richtung KNF möchten wir solchen unnötigen Klammern entfernen. 

Es ist daher wünschenswert, wenn ein Operator X einen Ausdruck als Kindknoten besitzt, in dem X ebenfalls der Operator ist, den Operator X im Kindknoten zu eliminieren und alle Kinder vom eliminierten Kindknoten an den verbleibenden Operatorknoten X zu hängen. Damit hätte man den Ausdruck vereinfacht, da die assoziative Klammerung wegfällt. Wir nennen dieses Vorgehen im Folgenden Operatorkompression.

Gegeben sei der ZQL-Parsebaum $B=(V,E)$. Es sei $child(v) = \{ w : w\in V \wedge (v,w)\in E\}$, also die Menge aller Kindknoten von $v$. Gibt es einen Knoten $w\in child(v)$ mit $v=w$, so wird Knoten $w$ eliminiert und alle Kindknoten von $w$ werden zu Kindknoten von $v$, also $child(v) = child(v) \cup child(w)$. 
$E=E\backslash \{ (w,x) : x\in child(w)\} \cup \{(v,x) x\in child(w)\}$ und $V=V\backslash \{w\}$.

Im Sinne des Vergleiches der Komplexität der Musterlösung mit der Komplexität der Lösung des Lernenden ist es sinnvoll zu speichern, ob und wie oft eine solche Operatorkompression durchgeführt werden musste.

\subsubsection{NOT auflösen}

Im nächsten Schritt möchten wir auftretende \verb|NOT| Operatoren entfernen. Dies geschieht indem der Operator \verb|NOT| im Parserbaum nach unten geschoben wird. Dabei werden die \textit{DE MORGAN} Regeln angewendet. 

\begin{tabular}{ll}
Eingabe: & Umwandlung Teil 1:\\
\verb|not ((a > 5)  and ((b > 5) or (c > 5)))| & \verb|(not(a > 5) or not((b > 5) or (c > 5)))|\\
\includegraphics[scale=0.5]{Bilder/not_graph1.png} & \includegraphics[scale=0.5]{Bilder/not_graph2.png}\\
\end{tabular}

\begin{tabular}{ll}
Umwandlung Teil 2: & Umwandlung Teil 3:\\
\verb|((a <= 5) or (not(b > 5) and not(c > 5)))| & \verb|((a <= 5) or ((b <= 5) and (c <= 5)))|\\
\includegraphics[scale=0.5]{Bilder/not_graph3.png} & \includegraphics[scale=0.5]{Bilder/not_graph4.png}\\
\end{tabular}


\subsubsection{Anwenden des Distributivgesetzes}

Im letzten Schritt haben wir die Formel \verb|((a <= 5) or ((b <= 5) and (c <= 5)))| erhalten. Durch Anwenden des Distributivgesetzes können wir diese Formel im letzten Schritt umformen zu: \verb|((a <= 5) or (b <= 5)) and ((a <= 5) or (c <= 5))|

\subsection{Ersetzung von syntaktischen Varianten}

Um eine Anfrage zu standardisieren müssen wir den syntaktischen Zucker entfernen. Dies geschieht, in dem man nur eine syntaktische Schreibweise anerkennt und alle anderen Schreibweisen werden in die zulässige umgewandelt. Zu erwähnen sind folgende Ersetzungen, die durchgeführt werden sollen um syntaktisch vielfältige, aber semantisch äquivalente Ausdrücke zu minimieren.

\begin{figure}[h]
\begin{tabular}{ccl}
\verb|A BETWEEN B AND C| & $\to$  & \verb|A >= B AND A <= C|\\
\verb|SELECT ALL| & $\to$ & \verb|SELECT|\\
\verb|ORDER BY VAR ASC| &  $\to$ & \verb|ORDER BY VAR|\\
\verb|A IN ('X', 'Y', 'Z')| & $\to$ & \verb|A = 'X' OR A = 'Y' OR A='Z'|\\
\verb|EXISTS (SELECT A,B,C ...)| & $\to$ & \verb|EXISTS (SELECT 1 ...)|\\
\end{tabular}
\caption{Entfernen von syntaktischen Varianten}
\end{figure}

\subsubsection*{Unteranfragen}

Es ist bekannt, dass sämtliche Typen von Unteranfragen eliminiert oder durch \verb|EXISTS| Anfragen ersetzt werden können. Streng genommen handelt es sich hier zwar um mehr als nur eine syntaktische Variante, aber dennoch wollen wir das Ersetzen von Unteranfragen in diesem Abschnitt betrachten.
\subsubsection*{Ersetzen von ALL}

\verb|... SAL >= ALL(1000, LOW_SAL)|  wird zu:\\
\verb|... SAL >= 1000 AND SAL >= LOW_SAL|

\subsubsection*{Ersetzen von ANY/SOME}

\begin{verbatim}
X.SAL >= ANY(SELECT Y.SAL FROM EMP Y 
             WHERE Y.JOB = 'CLERK')
\end{verbatim} zu
\begin{verbatim}
EXISTS(SELECT Y.SAL FROM EMP Y 
       WHERE Y.JOB = 'CLERK' 
       AND X.SAL >= Y.SAL)
\end{verbatim}

Befindet sich ein \verb|NOT| vor der Unteranfrage, so wird dieses nicht zur Unterabfrage ``durchgedrückt'' sondern bleibt davor.

\subsubsection{Ersetzen von IN}

Unter bestimmten Voraussetzungen kann jede IN-Unterabfrage in eine äquivalente EXISTS-Unteranfrage umgewandelt werden.

Wir wandeln IN-Unteranfragen wie die folgende:

\begin{lstlisting}[mathescape]
$t_1$ IN (SELECT $t_2$
      FROM $R_1\ X_1$, ..., $R_n\ X_n$
      WHERE $\varphi$)
\end{lstlisting}

unter -- noch zu erläuternden -- Vorrausetzungen um in:

\begin{lstlisting}[mathescape]
EXISTS (SELECT *
        FROM $R_1\ X_1$, ..., $R_n\ X_n$
        WHERE ($\varphi$) AND $t_1 = t_2$)
\end{lstlisting}

Folgende Voraussetzungen müssen erfüllt sein, damit diese Umwandlung angewendet werden kann.

\begin{itemize}
\item Alle Tupelvariablen, die in $t_1$ vorkommen, müssen sich unterscheiden von allen $X_i$. Erreicht wird dies ggf. durch Umbenennung der $X_i$, da diese ja nicht für die eigentliche (Ober)anfrage wichtig sind.
\item Wenn $t_1$ Attributreferenzen $A$ ohne Tupelvariable enthält, dann dürfen die $R_i$ kein Attribut $A$ haben. Erreicht wird dies, indem ggf. die Tupelvariable einführt.
\item Die Unteranfrage für $t_2$ darf keine Nullwerte liefern.
\end{itemize}

\subsubsection{andere Unteranfragen}

Ungewöhnliche Unterabfragen, wie z.B.: Unterabfragen unter \verb|FROM| werden hier nicht betrachtet. Im Allgemeinen werden solche Unterabfragen kaum gebraucht und machen die Anfrage meist nur viel komplexer als notwendig.

\subsubsection{JOINS}
Ein \verb|INNER JOIN| kann sowohl im \verb|FROM|, als auch im \verb|WHERE| Teil einer SQL-Anfrage formuliert werden. Damit Untersuchungen einheitlich geschehen können, formulieren wir solche JOINs im \verb|WHERE| Teil der SQL-Anfrage.

\begin{figure}[h]
Eingabe:\\
\verb|SELECT * FROM foo f INNER JOIN bar b ON f.id=b.id|\\

Umwandlung:\\
\verb|SELECT * FROM foo f, bar b WHERE f.id=b.id|\\
\caption{Umwandlung von INNER-JOIN}
\end{figure}

Bei Anwendung dieser Ersetzungsregeln, soll dem Lernenden ein klares Feedback gegeben werden. Es soll verdeutlicht werden, dass eine korrekte Anfrage dennoch Mängel aufweist, da unnötige Formulierungen benutzt wurden.

Es ist hier bereits möglich Terme, die nur aus numerischen Konstanten bestehen, zu Ersetzen durch das jeweilige Ergebnis. So könnten arithmetische Operationen bereits ausgeführt und Vergleiche, die nur aus numerischen Konstanten bestehen, durch entsprechende Wahrheitswerte ersetzt werden.

\subsection{JOIN Eliminierung}

In einem weiteren Zwischenschritt möchten wir gern einige unnötige \verb|JOINS| eliminieren.

\subsubsection{OUTER JOIN}

Betrachten wir eine SQL-Anfrage mit einem \verb|OUTER JOIN|. Befinden sich im \verb|SELECT|-Teil keine Attribute der \verb|JOIN|-Tabelle, so ist der \verb|JOIN| unnötig, weil Daten von der \verb|JOIN|-Tabelle ohnehin nicht ausgegeben werden. Betrachten wir dazu folgendes Beispiel:

\begin{figure}[h]\label{joinelem1}
\begin{lstlisting}[mathescape]
SELECT s.vorname, s.nachname 
FROM studenten s LEFT JOIN klausur k ON s.id = k.id 
WHERE k.wert <= 3
\end{lstlisting}
\caption{unnötiger OUTER JOIN}
\end{figure}

Im Beispiel in Abbildung \ref{joinelem1} sollen alle Studenten ausgegeben werden, die in einer Klausur die Note 3 oder besser bekommen haben. Da es sich hier um einen \verb|OUTER JOIN| handelt und wir nur Namen der Studenten ausgeben möchten, erhalten wir immer eine gesamte Liste der Studentennamen. Der Teil, der für den \verb|JOIN| interessant wäre, also Daten aus der Tabelle \verb|klausur k|, wird nicht ausgegeben. Damit wäre die folgende SQL-Anfrage mit der aus Abbildung \ref{joinelem1} äquivalent.

\begin{figure}[h]\label{joinelem2}
\begin{lstlisting}[mathescape]
SELECT s.vorname, s.nachname 
FROM studenten s 
\end{lstlisting}
\caption{unnötiger OUTER JOIN}
\end{figure}

Bei einem \verb|RIGHT JOIN| müsste man äquivalent überprüfen ob man Attribute im \verb|SELECT| Teil hat, die in der ``linken'' Tabelle sind und genauso verfahren.

Möchte man tatsächlich nur die Studenten ausgeben, die eine 3 oder besser geschrieben haben, müsste man einen \verb|INNER JOIN| verwenden.

\subsubsection{transitiv-implizierte INNER JOINs}

Wenn nur Schlüsselattribute einer Tupelvariable X benutzt werden und diese mit Fremdschlüsselattributen einer anderen Tupelvariable Y verglichen werden, dann ist X überflüssig.

In der Arbeit \cite{joinelem2} wird dazu ein Algorithmus angegeben, der im wesentliche oben genannte, überflüssige Tupelvariablen entfernt. Wir wandeln diesen Algorithmus leicht ab, erhalten aber die Grundidee.

Im ersten Schritt erstellen transitiv-abgeschlossene Äquivalenzklassen der Attribute, die im \verb|WHERE|-Teil vorkommen. Haben wir also $X.A=Y.B$ und $Y.B=Z.C$, so befinden sich alle drei Attribute in einer Äquivalenzklasse: $\{X.A,Y.B,Z.C\}$. Eine Äquivalenzklasse enthält nur Tupelvariablen und ist transitiv abgeschlossen über dem Operator $=$.
Dabei bemerken wir, dass Äquivalenzklassen der Mächtigkeit 1 einfache Vergleiche wie \verb|X.A Operator Konstante| sind. Klassen der Mächtigkeit 2 sind einfache, nicht transitive JOINs und Klassen der Mächtigkeit größer als 2 sind offensichtlich mehrfache, transitive JOINs.

Beispiel:
\begin{lstlisting}[mathescape]
SELECT t1.x 
FROM   test1 t1, test2 t2, test3 t3
WHERE  t1.x = t2.y
AND    t1.x = t3.z
AND    t1.y = t2.z
AND    t1.z > 3;
\end{lstlisting}

Bei diesem Beispiel erhalten wir die Äquivalenzklasse \verb|{t1.x, t2.y, t3.z}| sowie die Klassen \verb|{t1.y,t2.z}| und \verb|{t1.z}|.

Im nächsten Schritt gehen wir jede Äquivalenzklasse durch, die mindestens 3 Einträge haben. Für jeden Eintrag $e\in\mathit{Aequivalenzklasse}$ mit $e=T.A$ überprüfen wir, ob es andere Äquivalenklassen gibt, die nicht aus einem transitiven \verb|JOIN| entstanden sind (also weniger als 3 Einträge haben) und die ein Attribut der Tabelle $T$ enthalten. Ist dies nicht der Fall und die Tabelle $T$ kommt nicht im \verb|SELECT|-Teil vor, dann ist das Attribut $A$ in dem Vergleich, der durch die Äquivalenzklasse repräsentiert wird, unnötig und kann samt zugehöriger Bedingung im \verb|WHERE| Teil gestrichen werden. Hat die Äquivalenzklasse jetzt weniger als 3 Einträge dann wird die Arbeit an dieser Klasse abgebrochen.

Im letzten Schritt müssen wir noch überprüfen ob es Tupelvariablen im \verb|FROM|-Teil gibt, die aber nun nicht mehr im \verb|WHERE| Teil auftauchen. Ist dies der Fall, dann streichen wir diese aus dem \verb|FROM|-Teil.

NULL bearbeiten

Beispiel:
\begin{lstlisting}[mathescape]
SELECT ps.partkey, avg(ps.supplycost)
FROM   supplier s, partsupp ps, customer c, orders o
WHERE  s.suppkey = ps.suppkey 
AND    s.suppkey = c.custkey
AND    c.custkey = o_custkey
AND    o_totalprice >= 100;
\end{lstlisting}

Zunächst erzeugen wir Alle Äquivalenzklassen:
\begin{lstlisting}[mathescape]
Klassen = $\{ \{$s.suppkey, ps.suppkey, c.custkey,o.custkey $\}, \{$o.totalprice$\} \}$
\end{lstlisting}

Die zweite Menge interessiert uns nicht, da sie nicht mehr als zwei Elemente enthält. Wir betrachten daher nur die erste Menge. Wir untersuchen nun jedes einzelne Element auf seine Gültigkeit. \verb|s.suppkey| kommt in keiner anderen Äquivalenzklasse vor und die Tabelle \verb|supplier s| erscheint nicht im \verb|SELECT| Teil, daher streichen wir \verb|s.suppkey|. Das nächste Attribut \verb|ps.suppkey| kann nicht gestrichen werden, da die Tabelle \verb|partsupp ps| im \verb|SELECT|-Teil erscheint. \verb|c.custkey| kann wiederm gestrichen werden, da es wieder nicht in einer anderen Äquivalenklasse auftaucht und die Tabelle \verb|customer c| nicht im \verb|SELECT|-Teil auftaucht. Dahingegen ist \verb|o.custkey| nicht zu streichen, da die Tabelle \verb|orders o| in einer Äquivalenzklasse auftaucht, welche nicht zu einem \verb|JOIN| gehört (weil sie weniger als 3 Elemente beinhaltet). Nun können wir auch die Tabellen \verb|supplier s| und \verb|customer c| streichen, da keine Bedingung mehr Attribute aus diesen Tabellen enthält.

Daher erhalten wir nun folgende optimierte Anfrage:
\begin{lstlisting}[mathescape]
SELECT ps.partkey, avg(ps.supplycost)
FROM   partsupp ps, orders o
WHERE  o.custkey = ps.suppkey 
AND    o.totalprice >= 100;
\end{lstlisting}

PSEUDOCODE:

\begin{lstlisting}[mathescape]
Alle Attribute im WHERE Teil in Aequivalenklassen $E_i$ packen.
foreach $e\in E_i$ mit $\vert e\vert \ge 3$ do
  foreach $t.a\in e$ do
    if $\vert e\vert \ge 3$ and $t \notin $SELECT and 
    $\nexists b\in E_i$ mit $\vert b\vert < 3$ und b=t.*
      e = e - {t.a}
    end
  done
done

Streiche unnoetige Tabellen aus FROM Teil 
\end{lstlisting}

\subsection{Operatorenvielfalt}

Im folgenden Abschnitt soll geklärt werden wie mit verschiedenen Schreibweisen von ein und demselben Ausdruck umgegangen werden soll. Betrachtet man sich zum Beispiel: \verb|A > 5| ist dieser Ausdruck äquivalent mit \verb|5 < A|. Wenn wir wissen, dass \verb|A| ein ganzzahlige Variable ist, dann sind auch folgende Äquivalenzen wahr: \verb|A >= 6| so wie \verb|6 <= A|. Wir betrachten nun zwei verschiedene Ansätze um mit diesem Problem umzugehen. Ein Ansatz beschäftigt sich damit, alle implizierten Schreibweisen mit in die Formel aufzunehmen. Damit stellt man sicher, dass sich alle korrekten Schreibweisen einer Formel in der Anfrage befinden. Der zweite Ansatz beschäftigt sich damit, nur bestimmte Schreibweisen zuzulassen und alle anderen durch die zulässigen zu ersetzen.

Hinweis: Diesem Schritt geht eine Teilsortierung vor. Diese wird ebenfalls im Abschnitt >>Sortierung<< erwähnt. 

\subsubsection{Teilsortierung}

Wir betrachten Ausdrücke mit den Operatoren $\{>,<,\leq,\geq,=,+,\cdot\}$. Da es sich hier jeweils um binäre Operatoren handelt, sprechen wir -- im Sinne der Anordnung -- im Folgenden von einem linken und einem rechten Operanden. Ist einer der Operanden eine Variable, so wird diese links angeordnet. Sind beide Operanden Variablen, so werden sie lexikographisch-sortiert angeordnet. Operanden, die selbst wieder zusammengesetzte Ausdrücke sind, stehen rechts. Sind beide Operanden zusammengesetzte Ausdrücke, so steht der komplexere rechts und der weniger komplexe links. Ein Ausdruck $A$ ist komplexer als ein Ausdruck $B$, wenn der zugehörige Operatorbaum von $A$ tiefer ist als der Operatorbaum von $B$. Sind beide Ausdrücke gleich komplex, so wird die symmetrische Variante mit hinzugenommen. Wenn wir Operanden umsortieren bei denen der Operator $\in \{>,<,\leq,\geq\}$ ist, dann muss der jeweilige Operator auch umgedreht werden. Bei den restlichen Operatoren ist dies nicht der Fall, da diese symmetrisch sind.

\subsubsection{Hinzufügen implizierter Formeln}

Trifft man im Parserbaum auf eine Formel, zu der es mehrere äquivalente Formeln gibt, so ist ein Ansatz alle diese äquivalenten Formeln konjunktiv zu verknüpfen und mit in den Parserbaum aufzunehmen. Treffen wir also zum Beispiel auf folgenden Ausdruck \begin{verbatim}SELECT * FROM testtable WHERE A = B - C\end{verbatim}, so müssen wir auch alle äquivalenten Formeln mit aufnehmen. Daraus wird dann also der Ausdruck: \begin{verbatim}...WHERE A = B - C AND B = A + C AND C = B - A\end{verbatim}

Wie man bereits sieht, sind die hinzugefügten Formeln redundant und tragen nicht effizient zur Beschleunigung der Anfrage bei. Es soll hier lediglich sichergestellt werden, dass alle möglichen äquivalenten Formeln auftreten, da wir nicht wissen, was der Student für einen Repräsentanten der Formeln wählen wird. Weiterhin muss bemerkt werden, dass dadurch die gesamte SQL-Anfrage enorm aufgebläht wird. Es ist daher unbedingt wichtig, die Originalanfrage zu speichern. Weiterhin muss das Programm eine Verbindung zwischen den Formeln der Originalanfrage und den Formeln der veränderten, aufgeblähten Anfrage herstellen. Dem Lernenden soll in einem Feedback nur Fehler in der Originalanfrage aufgezeigt werden. Da intern aber mit der aufgeblähten Anfrage gearbeitet wird, muss beim Auftreten eines Fehlers oder Hinweises nachgeschlagen werden, von welchem Teil der Originalanfrage der Teil entstammt, der jetzt den Fehler auslöst.

Im Folgenden listen wir Mengen $M_i$ von Ausdrücken. Finden wir in der zu bearbeitenden SQL-Anfrage eine Formel $f$, die auf einen Ausdruck $a\in M_i$ passt, dann verknüpfen wir alle Ausdrücke $\{b : b\in M_i \wedge b \neq a\}$ konjunktiv mit $f$.

Im folgenden sind alle Variablennamen $A,B,C$ keine (komplexe) Ausrücke. Es handelt sich also jeweils um Blattknoten im Parserbaum. Ferner bezeichnen wir $X,Y$ als numerische Konstanten.\\

\begin{tabular}{ll}
$M_1$ & $\{\ A=B-C\ ,\ C=B-A\ ,\ B=A+C\ \}$\\
$M_2$ & $\{\ A=B\cdot C\ ,\ C=A / B\ ,\ B=A / C\ \}$\\
$M_3$ & $\{\ A>B-C\ ,\ C>B-A\ ,\ B<A+C\ \}$\\
$M_4$ & $\{\ A<B-C\ ,\ C<B-A\ ,\ B>A+C\ \}$\\
$M_5$ & $\{\ A>B, B<A \}$\\
$M_6$ & $\{\ A\geq B, B\leq A \}$\\
$M_7$ & $\{\ A>X, A\geq X+\mathit{adjust}(A) \}$\\
$M_8$ & $\{\ A<X, A\leq X-\mathit{adjust}(A) \}$\\

\end{tabular}

Beim Vergleich mit $>$ und $<$ ist es wichtig zu wissen, wie viel Nachkommastellen die numerischen Variablen $A$ und $B$ besitzen. Es sei $\mathit{places}(A)$ die Anzahl der Nachkommastellen der Zahl $A$. Dann bezeichnen wir mit $\mathit{adjust}(A) = 1 / (10^{\mathit{places}(A)})$, einen angepassten Wert, der sich nach der Stelligkeit der Variable A richtet.

Betrachten wir als Beispiel ein Attribut \verb|salery|, welches als \verb|NUMERIC(4,2)| definiert ist. Wir wissen also, dass \verb|salery| zwei Nachkommastellen hat. Betrachten wir nun die Aussage \verb|salery >= 5|.
Wir haben auf einer Seite eine Variable (\verb|salery|) und auf der anderen Seite eine numerische Konstante (\verb|5|). Dieses Muster passt also auf $M_7$ und auf $M_6$. In $M_7$ heißt es $A\geq X+\mathit{adjust}(A)$. Bezogen auf unser Beispiel ist \verb|A=salery| und \verb|x+adjust(salery)=5|. Wir berechnen also: 
$$\mathit{adjust}(\mathit{salery}) = 1 / (10^{2}) = 1/100 = 0,01$$
Wir erhalten also \verb|x=4,99|, weil \verb|x+0,01=5|. Somit ergänzen wir unsere Ausgangsformel \verb|salery >= 5| konjunktiv mit \verb|salery > 4,99|. Weiterhin muss jetzt wegen $M_6$ \verb|5 <= salery| und wegen $M_5$ \verb|4,99 < salery| hinzugefügt werden.

Finden wir Ausdrücke mit $>,<,\leq,\geq$, welche als Argumente Variablen oder Konstanten haben, so unterscheiden wir also grundsätzlich 3 Fälle.

Fall 1 $(M_5,M_6)$: Beide Operanden sind Variablen oder Konstanten. In diesem Fall ergänzen wir nur den jeweils symetrischen Operator. Da beide Operanden Variablen sind, macht es keinen Sinn jeweils $\leq,\geq$ oder $<,>$ zu ersetzen.

Fall 2 $(M_7,M_8)$: Einer der beiden Operanden ist eine numerische Konstante und der andere ist eine Variable. In diesem Fall fügen wir alle implizierten Gleichungen hinzu, also insbesondere die Operatoren $\leq,\geq,<,>$. Zu beachten ist hier, dass nicht nur Gleichungen der Form $A>X$ dazu führen, dass alle Ausdrücke von $M_7$ hinzugefügt werden. Auch wenn eine Gleichung der Form $Var1\geq 5.2$ auftaucht werden Ersetzungen durchgeführt. Diese Gleichung passt auf das Muster $A\geq X+\mathit{adjust}(A)$. Angenommen $Var1$ hat maximal eine Nachkommastelle, so würden dann folgende Gleichungen impliziert werden: $\{\ Var1>5.1\ ,\ 5.1<Var1\ ,\ 5.2 \leq Var1\ \}$.

Fall 3: Beide Operanden sind numerische Konstanten. In dem Fall wird die logische Aussage ausgewertet und durch ihren Wahrheitswert ersetzt [0,1].

Sind durch die hinzugefügten Terme nun arithmetische Ausdrücke entstanden, die nur noch numerische Konstanten enthalten, so werden diese Ausdrücke ausgewertet.

Dieser Teilschritt erfordert weiterhin eine Sortierung der einzelnen Terme.

\subsubsection{Beschränkung der Operatorenvielfalt}

Ein weiterer Ansatz das Problem der äquivalenten Formeln anzugehen ist es, bestimmte Operatoren zu >>verbieten<<. Das soll bedeuten, wir definieren verbotene Operatoren, welche am Ende der Umwandlungen nicht mehr in der SQL-Anfrage vorkommen dürfen. Dies wird erreicht, indem wir jeden verbotenen Operator umwandeln in einen nicht-verbotenen Operator. Das Prinzip ähnelt dem eben Vorgestelltem. Wir betrachten uns wieder die Mengen $M_i$. Des Weiteren hat jede Menge $M_i$ einen Repräsentanten $r(M_i)$. Finden wir nun in der zu bearbeitenden Anfrage eine Formel $f$, die auf eine der Ausdrücke $a\in M_i$ passt, so ersetzen wir $f$ mit $r(M_i)$. Folgende Tabelle soll die Mengen und deren Repräsentanten beschreiben.

Im folgenden sind alle Variablennamen $A,B,C$ keine (komplexe) Ausrücke. Es handelt sich also jeweils um Blattknoten im Parserbaum. Ferner bezeichnen wir $X,Y$ als numerische Konstanten.\\

\begin{tabular}{lll}
$i$ & $M_i$ & $r(M_i)$ \\
$1$ & $\{\ A=B-C\ ,\ C=B-A\ ,\ B=A+C\ \}$ & $B=A+C$\\
$2$ & $\{\ A=B\cdot C\ ,\ C=A / B\ ,\ B=A / C\ \}$ & $A=B\cdot C$\\
$4$ & $\{\ A>B-C\ ,\ C>B-A\ ,\ B<A+C\ \}$ & $A>B-C$ \\
$5$ & $\{\ A<B-C\ ,\ C<B-A\ ,\ B>A+C\ \}$ & $A<B-C$\\
$6$ & $\{\ A>B, B<A \}$ & $A>B$\\
$7$ & $\{\ A\geq B, B\leq A \}$ & $A\geq B$\\
$8$ & $\{\ A>X\ ,\ X<A,A\geq X+\mathit{adjust}(X)\ ,\ X\leq A - \mathit{adjust}(X)\ \}$ & $A>X$\\
\end{tabular}

Im Folgenden soll ein Beispiel die Prozedur verdeutlichen.\\

Es sei unsere Ausgangsanfrage: \begin{verbatim}SELECT * FROM testtable WHERE X = 6 - Y\end{verbatim}

Die Formel $X=6-Y$ finden wir in $M_1$ in Form von $A=B-C$. Wir ersetzen nun also $X=6-Y$ mit dem Repräsentanten von $M_1$, und wir bekommen: \begin{verbatim}SELECT * FROM testtable WHERE 6 = X + Y\end{verbatim}

\subsubsection{Diskussion der beiden Ansätze}

Ein wesentlicher Punkt beim Vergleich beider Ansätze ist der Aufwand bzw. die Laufzeit beider Ansätze. 

Betrachten wir zunächst den Ansatz des Hinzufügens von implizierten Formeln. Wir müssen in einer Tiefensuche jede Formel betrachten und mit allen Mengen $M_i$ abarbeiten. Finden wir in einer Menge ein Muster wieder, so wird unsere Formel künstlich aufgebläht. Wir haben also für das Suchen eine maximale Laufzeit von $\mathcal{O}(\mathit{\vert Formeln\vert \cdot max\{i : M_i\}})$. Das Einfügen der Formeln geschieht in konstanter Zeit $\mathcal{O}(1)$, da wir ja immer eine konstante Anzahl an Formeln ergänzen.

Beim anderen Ansatz werden bestimmte Operatoren verboten. Wir realisieren dieses Verbot wieder über eine Suche. Es muss auch hier jede Formel auf ein Muster in $M_i$ untersucht werden. Wir benötigen für das Suchen in diesem Ansatz also genau so viel Zeit, wie im ersten Ansatz. Auch das Ersetzen der Formeln hat keine Zeitersparnis gegenüber einem Hinzufügen von weiteren Formeln. Es muss bemerkt werden, dass in diesem Fall die Originalformel nicht weiter aufgebläht wird.

Da sich die Laufzeiten der beiden Varianten nicht unterscheiden, müssen andere Kriterien zum Vergleich herangezogen werden. Wichtig für Software ist nicht ausschließlich die Laufzeit, sondern auch die Wartbarkeit. Besonders bei Projekten, die im Rahmen einer Masterarbeit entstehen, ist es wahrscheinlich, dass der Autor sich später nicht mehr um das Projekt kümmern kann. Daher sollte man sich bei den hier vorliegenden Ansätzen fragen, welcher leichter wartbar und erweiterbar ist.

Muss das Programm erweitert werden und wir möchten den Ansatz des Hinzufügen implizierter Gleichungen verwenden, so muss lediglich eine weitere Menge $M_k$ erstellt werden. Der Algorithmus sucht automatisch, dann auch in dieser neuen Menge nach Mustern und würde alle anderen Elemente dieser Menge konjunktiv-verknüpft zur Formel hinzufügen. 

Bei der Verwendung von eingeschränkten Operatoren gestaltet sich dieser Ansatz bereits als schwierig. Hier muss man nicht nur die neue Menge $M_k$ angeben, sondern sich auch Gedanken machen, was ein geeigneter Repräsentant dieser Menge ist. Unter Umständen kann das Auswählen eines ungünstigen Repräsentanten zu unerwarteten Problemen, wie dem Verkomplizieren der Anfrage, führen.

Es bietet sich aus diesen Umständen eher an, das Hinzufügen von implizierten Gleichungen zu verwenden.

\subsection{Sortierung}

Im aktuell betrachteten Ansatz möchten wir zwei Anfragen dadurch vergleichen, dass wir sowohl die Musterlösung, als auch die Studentenlösung einer Standardisierung unterziehen. Ein ganz wesentlicher Aspekt dabei ist, die Art der Sortierung. Sind die ZQL-Parserbäume isomorph zueinander, dann lässt sich das leicht zeigen, in dem man beide nach gleichartigen Kriterien sortiert und dann einen direkten Abgleich vornimmt.

Dabei unterscheiden wir zwei Arten von Sortierung. Hat ein Operator als Operanden nur Ausdrücke und keine Konstanten oder Variablen dann sortieren wir die Kindknoten, welche jeweils wieder eigene Terme bilden.

Hat ein Operator als Operand mindestens eine Konstante oder Variable, so Sortieren wir das innere dieses Terms.

\subsubsection{Sortierung im Inneren der Terme}

Hat ein Operator \textit{OP1} als Kindknoten mindestens ein Blatt, dann werden die Kindknoten so sortiert, dass zunächst die Blattknoten (lexikographisch) und erst dann die Teilbäume erscheinen. Möglich wird dies, weil die Tabellen-Aliase in einem vorherigen Schritt bereits automatisch sortiert und benannt wurden. Bei symmetrischen Operatoren wie $=,  \textit{AND}, \textit{OR}$ können die Kindknoten einfach umgehangen/umsortiert werden. Bei Operatoren wie $\le,\ge$ ist es notwendig den Operator \textit{OP1} umzudrehen. Weil aber die Sortierung außerhalb von Termen auf den Operatoren basiert, ist es notwendig, die Sortierung im Inneren der Terme zuerst durchzuführen.

Dieser Schritt wurde bereits als Vorbereitung der Schritte >>Hinzufügen von implizierten Formeln<< und >>Operatorbeschränkung<< durchgeführt. 

\subsubsection{Sortierung von Termen}

Hat ein Operator \textit{OP1} als Kindknoten nur weitere Operatoren \textit{OP2,OP3}, dann muss anhand dieser Operatoren die Reihenfolge im Baum festegelegt werden. Dies geschieht, in dem wir uns einfach eine Reihenfolge der Operatoren ausdenken. Wir überlegen uns folgende Ordnung $order:\textit{Relation}\to\mathbb{N}$, in der eine Relation $r$ vor einer Relation $s$ im standardisierten Parserbaum erscheint, wenn $order(r) < order(s)$.

$order:$\\

\begin{tabular}{|llllllll|}
\hline
$r\in \textit{Relation}$ & $\le$ & $\ge$ & $>$ & $<$ & $=$ & IS NULL & IS NOT NULL  \\
$\textit{order}(r)$ & 1 & 2 & 3 & 4 & 5 & 6 & 7\\ 
\hline
\end{tabular}\\

Es sei $\textit{RT}(\textit{OP})$ der Teilbaum des SQL-Ausdruckes mit der Wurzel \textit{OP}. Wir bezeichnen mit $\textit{depth}(\textit{RT}(\textit{OP}))$ die Tiefe des Baumes $\textit{RT}(\textit{OP})$. Es seien $\textit{child}(\textit{OP}) = \{v_1,v_2,...,v_i,...,v_n\}$ die Kindknoten von $\textit{OP}$. Die korrespondierenden Teilbäume $\textit{RT}(v_1),\textit{RT}(v_2),...,\textit{RT}(v_i),...,\textit{RT}(v_n)$ sollen nun wie folgt angeordnet werden: Der Teilbaum $\textit{RT}(v_x)$ erscheint (bei einer fiktiven BFS) vor dem Teilbaum $\textit{RT}(v_y)$ genau dann, wenn $\textit{depth}(\textit{RT}(v_x)) <  \textit{depth}(\textit{RT}(v_y))$. Die Teilbäume werden also der Tiefe nach aufsteigend angeordnet. 

Wie bereits im Abschnitt >>Teilsortierung<< angedeutet, werden Teilbäume mit gleicher Tiefe nicht sortiert. In diesem Fall erzeugen wir einen Alternativbaum, indem wir die zwei betreffenden Teilbäume vertauschen. Mit diesem Alternativbaum wird dann, parallel zum bisherigen Baum, weiter verarbeitet. Am Ende muss die Musterlösung auch gegen alle Alternativlösungen geprüft werden.

Eine weitere Alternative zur Behandlung von Teilbäumen mit gleicher Tiefe, ist die Sortierung Anhand der Blattknoten. Dazu sammeln wir in einer Tiefensuche die Werte der Blattknoten und hängen sie in einem String zusammen. Es seien also die Knoten $V_{DFS} = \{v_1,v_2,...,v_k\}$ die Knoten, die in einer Tiefensuche eines (Teil)baumes entstehen. Wir bezeichnen den Blattknotenstring eines gewurzelten Baumes, mit dem Operator $O$ als Wurzel, mit: $$\mathit{leaf\_string}(RT(O)) = \mathit{val}(v_1) \oplus \mathit{val}(v_2) \oplus ... \mathit{val}(v_k)$$

Dabei bezeichnen wir mit $\mathit{val}(v_k)$, den Wert eines Blattknotens.

$$
val(v)=\begin{cases}
  \text{Variablenname},  & \text{wenn }v\text{ Variable,}\\
  \text{Wert}, & \text{wenn }v\text{ eine Konstante.}
\end{cases}
$$

Aus den Teilbäumen mit gleicher Tiefe, werden solche Strings erzeugt und diese dann verglichen.
Ist $\mathit{leaf\_string}(RT(O_1)) <  \mathit{leaf\_string}(RT(O_2))$, so wird der Teilbaum $RT(O_1)$ als erstes Kind im Baum auftauchen.

\begin{figure}[h]
\includegraphics[scale=0.5]{Bilder/same_depth1.png}         \includegraphics[scale=0.5]{Bilder/same_depth2.png}
\caption{Beispiel von Bäumen mit gleicher Tiefe}\label{bsp1}
\end{figure}

Wir sehen in Abbildung \ref{bsp1}, auf der linken Seite, einen zwei Teilbäume mit Wurzelknoten $>$. Wir bezeichen den linken Teilbaum mit $RT(>_l)$ und den rechten mit $RT(>_r)$. Somit ergeben sich $\mathit{leaf\_string}(RT(>_1)) = \mathit{val}(b) \oplus  \mathit{val}(3) = 'b3'$ und 
$\mathit{leaf\_string}(RT(>_2)) = \mathit{val}(b) \oplus  \mathit{val}(3) = 'a2'$. Wegen $\mathit{a2} < \mathit{b3}$ tauschen die Teilbäume ihre Position und ergeben das rechte Bild in Abbildung \ref{bsp1}. Da wir diesem Schritt die Tupelvariablen bereits durch automatische Variablennamen vereinheitlicht haben, kann der gesamte Name des Attributs samt Präfix der Tupelvariable als $\mathit{val()}$ angesehen werden.

\subsection{Abschluss}

Wir fassen die einzelnen Schritte noch einmal kurz zusammen. Zunächst haben wir den \verb|FROM| Teil der Anfrage vereinheitlicht, in dem wir einheitliche Tupelvariablen erzeugt haben, nachdem alle Tabellen im \verb|FROM| Teil lexikographisch sortiert wurden. Alle neu-erzeugten Variablennamen wurden im Rest der Anfrage korrekt eingesetzt bzw. ersetzt. Danach haben wir den \verb|WHERE| Teil bearbeitet. Wir haben zunächst unnötige Klammern entfernt und die Formeln in die KNF überführt. Danach haben wir einfache syntatkische Varianten ersetzt um eine einheitlichere Darstellung zu erhalten. Dazu gehörte es auch Unteranfragen aufzulösen oder, wenn nicht möglich, in eine \verb|EXISTS| Unteranfrage zu überführen. Danach haben wir innere Verbunde (JOIN), die im \verb|FROM| Teil formuliert wurden, in den \verb|WHERE| Teil überführt um anschließend unnötige äußere Verbunde und unnötige transitive, innere Verbunde  zu eliminieren. Eine der letzten Schritte war das Behandeln der Vielfalt der Operatoren. Dem ging zunächst eine Teilsortierung des Parserbaumes voraus. Mit einer der beiden vorgestellten Methoden haben wir nun alle implizierten Formeln hinzugefügt, oder die Operatoren gemäß ihrer Repräsentanten beschränkt. Schlussendlich haben wir den gesamten \verb|WHERE| Teil sortiert um eine Vereinheitlichung zu erreichen.

\section{Anpassung durch elementare Transformationen}

\section{weitere Betrachtungen}

Unabhängig von den bereits vorgestellten Ansätzen der >>Standardisierung<< und der >>Anpassung durch elementare Transformationen<< gibt es einige Umwandlungen, die entweder davor oder danach geschehen sollten. Diese Umwandlungen sollen dazu dienen dem Studenten ein Feedback zu geben. Das bedeutet, dass die Anfrage des Studenten richtig sein kann, allerdings unnötige oder unschöne Konstrukte enthält, welche die Anfrage unnötig kompliziert oder komplex machen.

Folgende verschiedene Komplexitätseinstufungen sollen eingeführt werden und auf jede Studentenanfrage angewendet werden.

\subsection{Anzahl atomarer Formeln}

Die Studentenanfrage enthält vor der Transformation durch unser Programm mehr atomare Formeln, als die Musterlösung, so wurden offensichtlich unnötige Formeln oder doppelte Formeln aufgeschrieben. Stellt unser Programm fest, dass beide Lösungen dennoch gleich sind, so muss dem Studenten mitgeteilt werden, das er redundante Formeln eingebaut hat, welche die Lösung unnötig verkomplizieren. 

\subsection{Anzahl der Operatorkompressionen}

Wie im vorherigen Abschnitt bereits erklärt ist der ZQL-Parserbaum nicht binär. Dadurch kann es durch zu vorsichtige Klammersetzung passieren, dass ein Teilbaum mit zwei Ebenen entsteht obwohl nur ein Operator beteiligt ist. Erklärt ist dies im Abschnitt >>Funktionsweise des Parsers<<. Die dort vorgestellte Operatorkompression ist ein Verfahren um unnötige Klammerungen zu entfernen. Ist die Gleichheit der Lösung des Studenten mit der Musterlösung durch unser Programm gezeigt, aber die Studentenlösung musste mehr Operatorkompressionen durchführen, so hat der Student unnötige Klammern gesetzt, welche die Lösung wiederum unnötig verkomplizieren. Dies muss ihm durch unser Programm mitgeteilt werden.

\subsection{unnötiges DISTINCT}

Eine interessante Frage ist, ob ein \verb|DISTINCT| wirklich notwendig ist. Dies ist natürlich wichtig für den Vergleich zweier SQL-Anfragen. In \cite{brass2} wurde im Rahmen des SQLLint Projektes bereits in den Prototypen ein Checker eingebaut, der prüft ob \verb|DISTINCT| wirklich notwendig ist. Aber auch im Rahmen dieser Arbeit ist es notwendig zu wissen, ob das \verb|DISTINCT| notwendig ist. 

Auf den ersten Blick reicht es aus zu prüfen, ob die Musterlösung ein \verb|DISTINCT| enthält. Ist dies ist der Fall, so muss die Lösung des Lernenden dieses offensichtlich auch enthalten. Allerdings setzt dieser Denkansatz voraus, dass die eingetragene Musterlösung stets perfekt ist. Um Fehler vorzubeugen ist es besser, alle Anfragen auf unnötige \verb|DISTINCT| zu prüfen. So kann dem Korrektor beim Eintragen der Musterlösung bereits angezeigt werden, dass sein angegebenes \verb|DISTINCT| unnötig ist oder ob ein \verb|DISTINCT| notwendig wäre um Duplikate zu eliminieren. Auch wenn man sich weg bewegt vom Modell der Musterlösung und dem Vergleich mit dem Lernenden, ist dieser Check durchaus wichtig. Im Folgenden stellen wir daher einen Algorithmus vor, der erkennt ob die Lösung Duplikate enthalten kann oder nicht.

\subsection{Algorithmus aus \cite{sql1folien}}

Es sei unsere SQL-Anfrage der Form:

\begin{lstlisting}[mathescape]
SELECT $t_1$, ..., $t_k$
FROM $R_1\ X_1$, ..., $R_n\ X_n$
WHERE $\varphi$
\end{lstlisting}

Es sei $X=\{X_1, ..., X_n\}$ die Menge aller Tupelvariablen. Es sei $G=\{G_1, ..., G_m\}$ die Menge aller \verb|GROUP BY| Spalten.

Die Einzelnen Attribute $t_i$ haben die Form $t = X.k$. Dabei ist $X$ eine Tupelvariable und $k$ ein Attribut. Wir bezeichnen die Menge aller $t_i$ mit $\mathcal{K}=\{t_1,...,t_k\}$.

\begin{lstlisting}[mathescape]
$\mathcal{K}$ = $\mathcal{K}\ \cup$ A, wenn $A=c\in$ WHERE-Bedingung
do 
  $\mathcal{K'}$ = $\mathcal{K}$
  $\mathcal{K'}$ = $\mathcal{K'}\ \cup$ A, wenn $A=B\in$ WHERE-Bedingung und $B\in\mathcal{K}$
  $\mathcal{K'}$ = $\mathcal{K'}\ \cup$ S mit $S=\{b\in X\}$, wenn $t\in \mathcal{K'}$ ein Schluessel ist und t=X.k
while ($\mathcal{K}\ \neq\ \mathcal{K'}$)

if Anfrage hat GROUP BY Statement:
  foreach $x\in X$ do
    if not ($\exists k\in \mathcal{K'}$ mit $k$ ist Schluessel von $x$)
      break and answer NO
    endif
  done

if not Anfrage hat GROUP BY Statement:
  foreach $g\in G$ do
    if $g\notin \mathcal{K'}$
      break and answer NO
    endif
  done

answer YES
\end{lstlisting}



%Parser, Java Version, Eigener Parser, Anbindung an fremde DB Oracle, Mysql, etc.

\chapter{Praktische Umsetzung}
\label{chap:praxis}
\section{Anforderungen}

Das zu entwickelnde Programm wird online per Java Servlets zur Verfügung gestellt. Als Webserver wird der Tomcat eingesetzt. Folgende Funktionalitäten sollen umgesetzt weden:

\begin{itemize}
\item Loginsystem für Lernenden und Betreuer
\item Grafische Oberfläche
\item Eintragen von neuen Übungsaufgaben bestehend aus:
	\begin{itemize}
	\item Sachtext /Aufgabenstellung
	\item Musterlösung(en)
	\item Angabe von realen Datenbanken mit Testdaten
	\item Einteilung in Kategorie
	\end{itemize}
\item Löschen und Bearbeiten von Übungsaufgaben
\item Erstellen, Bearbeiten und Löschen von Kategorien
\item Lösen von Übungsaufgaben durch Angabe einer SQL-Anfrage
\item Tracking von Lösungsversuchen (Versuch, AufgabenID, Timestamp, UserID)
\item Tracking von Usern und letzten Aktivitäten
\item Tracking von Fehlern/ Fehlversuchen pro Aufgabe/User
\end{itemize}

\chapter{Schlussteil}
\label{chap:ausblick}

\section{Zusammenfassung}

Die Aufgabenstellung für diese Arbeit war es, Methoden zu entwickeln mit denen man in der Lage ist zwei SQL-Anfragen miteinander zu vergleichen. Dabei sollte herausgefunden werden, ob die beiden Anfragen semantisch äquivalent sind und damit immer die gleichen Ergebnistupel produzieren. Entwickelte Methoden sollten dann in einem Programm realisiert werden, welches potentiell in der Lehre einsetzbar wäre. 

Zu diesem Zweck haben wir zunächst untersucht, was wir genau von unserem Programm erwarten. Wir haben festgestellt, dass semantisch-äquivalente Lösungen meist syntaktisch ähnlich sind. Daher haben wir uns überlegt, die SQL-Anfragen aneinander anzugleichen, indem wir legale Umformungen an beiden SQL-Anfragen durchführen. Motiviert von anderen Problemen in der Informatik, in denen eine Vorsortierung ein Problem oft vereinfachen kann, haben wir uns dazu entschlossen zunächst beide SQL-Anfragen nach festen Regeln zu standardisieren. Ziel dabei sollte sein, dass alle semantisch-äquivalenten SQL-Anfragen nach der Standardisierung auch syntaktisch gleich sind. Wenn zwei SQL-Anfragen nach der Standardisierung syntaktisch gleich sind, dann würden sie auch semantisch gleich sein. Dieser Schritt prüft also eine hinreichende Bedingung für semantische Äquivalenz.

Da es Anfragen gibt, die semantisch äquivalent sind, aber sich nicht aneinander angleichen lassen, haben wir einen zweiten Schritt entwickelt. Wir führen beide SQL-Anfragen auf Datenbanken mit echten Daten aus und vergleichen die Ergebnistupel. Sind diese nicht identisch, so verstößt dies gegen eine notwendige Bedingung: ``Auf allen Datenbankzuständen müssen zwei semantisch äquivalente SQL-Anfragen stets das gleiche Ergebnis liefern.'' 

Nach eingehender Analyse der bereits existierenden Lernplattformen zum Thema SQL, konnten wir feststellen, dass dieser Ansatz noch nicht realisiert worden ist. Wir haben Lernplattformen gesehen, die sich mit unserer Idee überschneiden, sie aber nie in dieser Form umsetzen. Nach dem Sichten von existierenden Lernplattformen haben wir den Fokus dieser Arbeit auf das Angleichen von SQL-Anfragen gelegt und verzichteten dafür, zum großen Teil, auf semantische Fehlermeldungen. 

Bevor wir mit der theoretischen Entwicklung unserer Methoden begonnen haben, untersuchten wir genau die zur Verfügung stehende Software. Dabei stand insbesondere der verwendete SQL-Parser 'ZQL' im Fokus der Betrachtungen. Nach einer eingehenden Analyse haben wir festgestellt, dass dieser einfach zu Verwenden und zu Erweitern ist, was ihn für unsere Zwecke attraktiv gemacht hat. Bei der Analyse haben wir auch festgestellt, dass der Parser Schwächen hat, die es vorerst verhindern einige unserer Konzepte in einem praktischen Programm umzusetzen.

In Kapitel \ref{chap:theorie} haben wir dann unsere Ideen aus der Vorbetrachtung in konkrete Methoden entwickelt. Wir haben uns überlegt, welche gleichen Konzepte in SQL in verschiedener Art und Weise formulierbar sind. Für Unterabfragen und Verbunde konnten wir so Methoden entwickeln, die die Vielfältigkeit der Formulierung dieser einschränken. Weiterhin haben wir syntaktische Variationen vereinheitlicht und uns mit der Frage beschäftigt wie man der Operatorenvielfalt in einer SQL-Anfrage begegnen kann. Schlussendlich entwickelten wir eine Sortiermethode, die es uns erlaubt eine feste Ordnung in einem Operatorbaum, der den \verb|WHERE|-Ausdruck einer SQL-Anfrage darstellt, zu etablieren. Aufgrund der Beschränktheit durch den Parser war es uns nicht möglich alle entwickelten Methoden und Betrachtungen in unsere Anwendung umzusetzen. Weiterhin haben wir untersucht was für Probleme auftreten, wenn wir SQL-Anfragen auf externen Datenbanken ausführen, um die notwendige Bedingung der Äquivalenz zu prüfen.

Nachdem die entwickelten Methoden mit Java und JSP implementiert wurden, haben wir die Programmstruktur im Kapitel \ref{chap:praxis} eingehend erläutert. Die Anwendung wurde im Wesentlichen in drei Hauptprozesse aufgeteilt, die mit Hilfe der JSP gesteuert und dem Nutzer per Web-UI zur Verfügung gestellt werden. 

Es wurde erfolgreich eine Anwendung entworfen, die theoretische Betrachtungen und, in dieser Arbeit, entwickelte Methoden umsetzt. Dabei werden, wie gefordert, zwei SQL-Anfragen miteinander verglichen. Im konkreten Fall handelt es sich dabei um eine Musterlösung und eine Lösung eines Lernenden. Es wird geprüft, ob die beiden semantisch Äquivalent sind. Die Anwendung kann als Grundstein für eine Lernplattform dienen, in der es leicht ist neue Aufgaben einzupflegen und der Lernende ein umfassenderes Feedback erhält, als es vom normalen SQL-Parser eines DBMS möglich ist.

\section{Ausblick}

Einige der entwickelten Methoden aus Kapitel \ref{chap:theorie} konnte nicht umgesetzt werden. Die Ursachen dafür liegen hauptsächlich bei den Schwächen des verwendeten Parsers. Dies sollte für zukünftige Arbeiten aber nur minimale Probleme mit sich bringen, da der Parser quelloffen ist und unter der GNU GPLv3 steht. Damit ist er leicht anpassbar und kann auch offiziell wiederverwendet werden. Die nicht implementierten Methoden sind dennoch ausführlich behandelt, sodass ein nachträgliches Implementieren einfach möglich ist. Eine große Schwäche des Parsers ist es, dass die \verb|FROM|-Klausel nur als Liste von Relationen geparst wird. Dadurch verhindert es der Parser Verbunde und Unterabfragen in \verb|FROM| zu formulieren. Während Unterabfragen unter \verb|FROM| recht selten sind, sind aber Verbunde innerhalb der \verb|FROM|-Klausel durchaus üblich. Würde der Parser um diese Funktionen erweitert werden, so ist es leicht möglich eine Vielzahl unserer Verbundsüberlegungen aus Abschnitt \ref{subsec:joins} zu übernehmen.

Weiterhin gibt es noch Probleme bei der Standardisierung, die wir nicht ausführlich betrachtet haben. So ist weiterhin offen, wie genau man mit komplexeren arithmetischen Ausdrücken in einer Anfrage umgeht. Wir haben zwar einige Ideen aufgezeigt, diese sind aber nicht umfangreich genug, um sie in einem Algorithmus zu manifestieren. Weiterhin haben wir uns von Anfang an auf \verb|SELECT|-Anfragen beschränkt. Um alle SQL-Anfragen abzudecken muss noch untersucht werden, wie mit den übrigen Arten von SQL-Anfragen umgegangen werden muss, damit diese vergleichbar sind. 

Nützlich für unsere Anwendung wäre es außerdem, wenn wir Methoden von anderen Lernplattformen übernehmen und in unsere integrieren. Denkbar wäre es die Aufgaben mit einer Schwierigkeitsstufe zu versehen, damit man ein 'Matchmaking'-System so etablieren kann, dass der Student weder gelangweilt noch überfordert ist. Weiterhin ist denkbar viel mehr semantic-checks einzubauen, wie \mbox{z. B.} eine ausführliche Analyse des Wertebereichs von Attributen. Damit könnte man viele inkonsistente Anfragen bereits vor der Standardisierung erkennen. Dies würde dem Lernenden, unabhängig von der Musterlösung, mehr Feedback zu seiner SQL-Anfrage geben. 

Unter Umständen haben wir den Fall, dass wir die Äquivalenz zweier Anfragen weder bestätigen, noch ablehnen können. Um die Leistung unserer Anwendung zu steigern gilt es, die Häufigkeit solcher Fälle zu verringern. Sind also Verbesserungen implementiert worden, so macht es Sinn Testreihen mit Studenten durchzuführen und zu messen, ob die Häufigkeit dieser ungünstigen Fälle zurückgegangen ist.



%\bibliographystyle{plain}
\printbibliography

\chapter{Anhang}
\section{Benutzerhandbuch} 

Das Benutzerhandbuch ist sowohl für Linux als auch für Windows geeignet. Befolgen Sie bitte strikt die Installationsanweisungen, um einen korrekten Ablauf der Software zu gewährleisten.\\

Entpacken Sie das Paket sqlequalizer.zip in ein beliebiges Verzeichnis. Wir bezeichnen dieses Verzeichnis im Folgenden als \textbf{basedir}.

\subsection{Installation}

\subsubsection{Vorbereitung}

Der SQL-Equalizer benötigt verschiedene Software, um korrekt zu funktionieren. Installieren Sie bitte zunächst einen aktuellen Tomcat-Server. Die Software und Installationsanleitung für einen Tomcat-Server finden Sie unter \url{http://tomcat.apache.org/}.\\

Um die Software zu übersetzen benötigen Sie außerdem ein Java Development Kit, zu finden unter \url{http://www.oracle.com/technetwork/java/javase/downloads/index.html}. Um einen automatisierten Ablauf zu gewährleisten, installieren Sie bitte Apache Ant, zu finden unter \url{http://ant.apache.org/}.\\

\subsubsection{Installation der Datenbank}

Der SQL-Equalizer unterstützt MySQL, PostgreSQL und Oracle DB. Wir gehen davon aus, dass die von Ihnen präferierte Datenbank bereits zur Verfügung steht.\\

Um die interne Datenbank für den SQL-Equalizer zu installieren, verwenden Sie die Datei\\\verb|sqlequalizer.sql| im basedir-Verzeichnis. Achten Sie darauf, dass die Datei MySQL kompatibel ist. Wenn Sie ein anderes DBMS verwenden wollen, ändern Sie die Datei nach ihren Wünschen ab. Achten Sie dabei auf die Typen \verb|DATETIME| und \verb|TEXT|.\\

Öffnen Sie die Datei \verb|Connector.java|, die Sie unter \url{BASEDIR/src/de/unihalle/sqlequalizer} finden. Ab Zeile 44 finden Sie jeweils kommentierte Bereiche, die für PostgreSQL und Oracle DB stehen. Wenn Sie MySQL verwenden wollen, dann muss nichts geändert werden. Möchten Sie eine der anderen Datenbanken verwenden, so kommentieren Sie die entsprechenden Bereiche aus.\\

\subsubsection{Installation der Software}

Wir nennen das Verzeichnis, in dem der Tomcat-Server installiert ist, \textbf{tomcatdir}.
Öffnen Sie die Datei \verb|build.xml| im basedir-Verzeichnis und ändern Sie die Zeile 4. Tragen Sie dort unter \verb|value| das Verzeichnis \verb|tomcatdir/webapps| ein. Achten Sie darauf, dass die Zeichenkette tomcatdir durch das korrekte Verzeichnis ersetzt worden ist.\\

Öffnen Sie ein Terminal.\\
Windows: Drücken Sie die Windows-Taste und 'r' gleichzeitig. Tippen Sie das Wort 'cmd' ein und drücken Enter.\\
Linux: Benutzen Sie die Shell.\\

Navigieren Sie in das basedir-Verzeichnis durch Verwendung des 'cd'-Kommandos. Dort angekommen, tippen Sie 'ant' in das Terminal ein und bestätigen den Befehl mit Enter. Haben Sie Apache Ant korrekt installiert, dann startet nun der Installationsprozess. Am Ende des Prozesses sollte eine sqlequalizer.war-Datei erstellt worden sein. Diese wurde auch automatisch in das tomcatdir/webapps Verzeichnis kopiert, so dass sie bereits 'deployed' ist. 

Navigieren sie mit ihrem Webbrowser folgende URL an: \url{http://hostname:8080/sqlequalizer/}. Sie sehen jetzt einen Loginbildschirm. Loggen Sie sich mit den Daten: admin/secure1234 ein. Wenn Sie sich einloggen konnten, ist die Softwareinstallation abgeschlossen.

\subsection{Admin Control Panel}

Im Folgenden wird beschrieben, wie neue Aufgaben in die Datenbank eingepflegt werden können. Dabei sind die einzelnen Schritte bereits in der korrekten Reihenfolge aufgeführt. Eine Übersicht über das Admin Control Panel sehen wir in Abbildung \ref{fig:acp}.

\begin{figure}[H]
\centering
\includegraphics[scale=0.51]{Bilder/screen_acp_2.png}
\caption{Übersicht über das Admin Control Panel}
\label{fig:acp}
\end{figure}

\subsection{Datenbankschema erstellen}

Navigieren Sie auf das Admin Control Panel. Betätigen Sie den Link 'add database schema'. Daraufhin erhält die Tabelle ein weiteres Schema, was mit \verb|Empty, Empty| markiert ist. Klicken Sie auf 'edit', um das Schema anzupassen. Die erste Spalte bezeichnet den Namen des Datenbankschemas. Damit lässt es sich bei der Zuordnung zu einer Aufgabe leichter wiederfinden. In der zweiten Spalte befindet sich ein Eingabefeld. Tragen Sie hier das Schema ein. Verwenden Sie Dabei \verb|CREATE TABLE|-Anweisungen, die mit Semikolon (;) abgetrennt sind. Betätigen Sie danach den 'save'-Knopf.

\subsection{Externe Datenbank anbinden}

Die SQL-Anfragen des Lernenden und die Musterlösung werden auf externen Datenbanken zur Kontrolle ausgeführt. Um eine externe Datenbank anzubinden, navigieren Sie im Admin Control Panel auf den Link 'external databases'. Klicken Sie nun auf 'add external database' und anschließend auf 'edit'.  Tragen Sie Daten entsprechend der folgenden Tabelle in die Felder ein.

\begin{tabular}{|l|p{14cm}|}\hline
Spatel 1 & Tragen Sie hier den Hostnamen und den Port des DBMS getrennt mit Doppelpunkt (:) ein. Der Port muss auch angegeben werden, wenn es der Standard-Port ist.\\\hline
Spatel 2 & Hier wird der Name der Datenbank eingetragen.\\\hline
Spatel 3 & Tragen Sie hier ein, was sie für ein DBMS benutzen. Zur Auswahl stehen MySQL, PostgreSQL und Oracle DB.\\\hline
Spatel 4 & Hier tragen Sie den Benutzernamen für den Zugriff auf die Datenbank ein.\\\hline
Spatel 5 & Hier tragen Sie das Passwort für den Zugriff auf die Datenbank ein.\\\hline
\end{tabular}



Es muss darauf geachtet werden, dass die externe Datenbank bereits Tabellen mit Daten enthält. Der SQL-Equalizer legt auf externen Datenbanken keine Tabellen oder Datensätze an.

\subsection{Aufgabe erstellen}

Um eine neue Aufgabe einzupflegen, navigieren Sie im Admin Control Panel auf den Link 'add task' und anschließend auf 'edit'. Sie sehen nun eine neue Eingabemaske, wie sie auch in Abbildung \ref{fig:acp2} zu sehen ist. Tragen Sie die Daten entsprechend der folgenden Tabelle ein.

\begin{figure}
\centering
\includegraphics[scale=0.61]{Bilder/screen_acp_1.png}
\caption{Erstellen einer Aufgabe im ACP}
\label{fig:acp2}
\end{figure}

\begin{tabular}{|l|p{12cm}|}\hline
description & Tragen Sie hier die Aufgabenstellung als Textform ein. Beschreiben Sie, welche Spalten welcher Tabelle in was für einer Sortierung (falls gewünscht) ausgegeben werden sollen, so dass der Student möglichst genau weiß, was er zu tun hat.\\\hline
sample solutions & Tragen Sie hier die Musterlösung in Form einer SQL-Anfrage ein. Der Student kann diese Anfrage nicht sehen. Möchten Sie mehrere Musterlösungen verwenden, weil sie sich strukturell stark unterscheiden, so verwenden Sie den Link 'add another sample solution', um mehrere Eingabefelder zu erzeugen.\\\hline
database schema & Verwenden Sie das dropdown-Menü, um ihr erstelltes Datenbankschema hier anzugeben.\\\hline
respect column order & Haken Sie das Kästchen an, wenn die Spalten genau die angegebene Reihenfolge einhalten sollen.\\\hline
external database & Markieren Sie hier die externen Datenbanken, auf denen die Musterlösung und die Lösung des Lernenden ausgeführt werden sollen.\\\hline
\end{tabular}

Nachdem Sie den 'save'-Knopf betätigt haben, ist die Aufgabe im System eingepflegt.


\subsection{Handbuch für Anwender}

Der SQL-Equalizer ermöglicht es, das formulieren von SQL-Anfragen zu üben. Dabei melden Sie sich als Nutzer mit ihrem Browser beim SQL-Equalizer an. Sie können dann Aufgaben lösen und erhalten vom System Rückmeldung, ob ihre Lösung korrekt war oder nicht. Dabei zeigt ihnen der SQL-Equalizer in welchem Bereich ihrer SQL-Anfrage Probleme aufgetaucht sind. Sie haben dann die Möglichkeit erneute Lösungsversuche einzureichen. 

\subsubsection{Übersicht}

Nachdem Sie sich angemeldet haben, sehen Sie den Startbildschirm. Dieser ist in drei Bereiche geteilt: Im ersten Bereich sehen Sie ihre fünf letzten Einsendungen. In einer Tabelle wird Ihnen dabei das Datum der Einsendung (time), die Aufgabennummer (task), ihr SQL-Statement (sql statement) sowie die Korrektheit der Lösung (correct?) angezeigt.

Im Feld 'correct?' gibt es drei mögliche Werte: Konnte ihr SQL-Statement durch das Standardisierungsverfahren an die Musterlösung angepasst werden, so erhalten Sie dort die Ausgabe \textbf{yes}. Ist dies nicht gelungen und ihre Lösung liefert beim Ausführen auf einer externen Datenbank mit realen Daten nicht die gleichen Ergebnisse wie die Musterlösung, so erhalten Sie die Ausgabe \textbf{no}. Ansonsten erhalten Sie die Antwort \textbf{unknown}. Dabei ist zu beachten, dass ihre Lösung im dritten Fall dennoch korrekt sein kann. Dem Dozenten werden solche Lösungen angezeigt, damit dieser dann entscheiden kann, ob ihre Lösung korrekt ist oder nicht.

\begin{figure}[H]
\centering
\includegraphics[scale=0.7]{Bilder/screen_user_1.png}
\caption{Informationen über die letzten fünf Lösungsversuche}
\end{figure}

Im nächsten Bereich, sehen Sie eine Statistik über alle Aufgaben. Dabei wird in einer Tabelle angezeigt, wie viele Lösungsversuche sie pro Aufgabe bisher benötigt haben (\#solutions). Weiterhin wird angezeigt, wie viele von diesen Lösungsversuchen korrekt waren (correct). Eine Erfolgsrate (ratio) und der Zeitpunkt des letzten Lösungsversuches vervollständigen die Anzeige.

\begin{figure}[H]
\centering
\includegraphics[scale=0.7]{Bilder/screen_user_2.png}
\caption{Statistik über alle Aufgaben}
\end{figure}

Im dritten und letzten Bereich, finden Sie eine Übersicht über alle bisher ungelösten Aufgaben. Dort werden Ihnen die Anzahl der Lösungsversuche sowie ein Hyperlink zur Aufgabe angezeigt.

\begin{figure}[H]
\centering
\includegraphics[scale=0.7]{Bilder/screen_user_3.png}
\caption{Ungelöste Aufgaben}
\end{figure}

\subsubsection{Einsenden einer Lösung}

Nachdem eine Aufgabe zum Lösen ausgewählt wurde, öffnet sich eine Eingabemaske. Oben sehen Sie die Aufgabenstellung in Textform. Diese enthält Informationen, welche Spalten im Ergebnis angezeigt werden sollen. Darunter befindet sich das Datenbankschema in Form von \verb|CREATE TABLE|-Anweisungen. Diesem Schema kann entnommen werden, welche Spalten in welcher Art und Weise definiert sind. Darunter befindet sich ein Eingabefeld, in das die Lösung eingetragen werden kann. Der Aufbau der Eingabemaske ist in Abbildung \ref{fig:maske1} zu sehen.

\begin{figure}[h]
\centering
\includegraphics[scale=0.6]{Bilder/screen_user_4.png}
\caption{Eingabemaske}
\label{fig:maske1}
\end{figure}


Nachdem Sie die Lösung abgeschickt haben, gibt Ihnen der SQL-Equalizer ein detailliertes Feedback dazu. Zunächst wird die geparste Eingabe sowie - auf Wunsch - die standardisierte Eingabe angezeigt.

\begin{figure}[H]
\centering
\includegraphics[scale=0.6]{Bilder/screen_user_5.png}
\caption{Geparste Eingabe}
\end{figure}

Darunter befinden sich die Kommentare des SQL-Equalizer in drei Abschnitten. Im ersten Abschnitt wird angezeigt, ob die eingesandte Lösung durch Standardisierung mit der Musterlösung übereinstimmt. Ist dies der Fall, werden alle weiteren Ausgaben unterbunden und Sie erhalten die Meldung, dass ihre Lösung korrekt ist.

Im zweiten Abschnitt werden die zwei Ergebnismengen, die von der eingesandten und der Musterlösung erzeugt worden sind, verglichen. Geben beide Anfragen die gleichen Ergebnisse zurück, wird Ihnen dies angezeigt, zusammen mit der Meldung, dass der SQL-Equalizer nicht sagen kann, ob ihre Lösung korrekt ist oder nicht. Unterscheiden sich die Ergebnismengen jedoch, wird Ihnen angezeigt, dass ihr Lösungsversuch nicht korrekt war.

Im letzten Abschnitt weist Sie der SQL-Equalizer auf Abweichungen in ihrer Lösung hin. Dabei vergleicht der SQL-Equalizer die einzelnen Teile ihrer Anfrage mit den entsprechenden Teilen der Musterlösung. Auch andere Eigenheiten ihrer beiden Anfragen werden verglichen. Das Ergebnis dieser Anzeige wird Ihnen dann in diesem Abschnitt angezeigt.

Ein Beispiel für eine solche Ausgabe sehen Sie in Abbildung \ref{fig:screen_user_1}

\begin{figure}[H]
\centering
\includegraphics[scale=0.61]{Bilder/screen_user_6.png}
\caption{Rückmeldung vom SQL-Equalizer}
\label{fig:screen_user_1}
\end{figure}

Hiermit versichere ich, dass ich die Abschlussarbeit bzw. den entsprechend gekennzeichneten Anteil der Abschlussarbeit selbständig verfasst, einmalig eingereicht und keine anderen als die angegebenen Quellen und Hilfsmittel einschließlich der angegebenen oder beschriebenen Software benutzt habe. Die den benutzten Werken bzw. Quellen wörtlich oder sinngemäß entnommenen Stellen habe ich als solche kenntlich gemacht.


\vspace{10ex}

Halle (Saale), 27. August 2013 


\end{document}
