In dieser Arbeit haben wir die Operation Fortsetzung $\ff$ untersucht. Diese wurde in der Arbeit [St87] eingef"uhrt und beschreibt den Durchschnitt zweier $\delta$-Limites.
Wir haben die algebraische Struktur $({2^X}^*,\ff)$ untersucht und festgestellt, dass es sich aufgrund fehlender Assoziativit"at weder um eine Gruppe noch um eine Halbgruppe handeln kann. Wir haben weiterhin gezeigt, dass die Struktur auch nicht kommutativ ist, aber Null- sowie Einselement besitzt. 
\\\\Anschlie"send untersuchten wir die Stabilit"at der Operation Fortsetzung bez"uglich $\cap,\cup$ und Konkatenation und die 
Monotonie bez"uglich $\subseteq$ im Vorder- sowie Hinterglied. Die Ergebnisse sind im Abschnitt 3 festgehalten.
In der Arbeit [St97] wurden Sprachen mit der Struktur $W\cdot X^*$ als offen bezeichnet und solche mit der Struktur $\pref(L)$ als geschlossen. Daher wurde die Operation Fortsetzung untersucht, wenn eine der beiden Operanden eine solche Struktur hat.
Dabei haben wir festgestellt, dass sich f"ur $W\ff\pref(V)$ und $V\cdot X^*\ff W$ die Fortsetzung auf den Durchschnitt vereinfachen l"asst.
\\\\Abschlie"send haben wir die Abgeschlossenheitseigenschaften der Operation Fortsetzung für die Klassen der CHOMSKY-Hierarchie untersucht und konnten erhaltende Eigenschaften feststellen, welche wir im Abschnitt 5 festgehalten haben.

