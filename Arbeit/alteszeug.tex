\section{Einleitung}

Die Idee, SQL-Anfragen von Lernenden automatisch zu kontrollieren, ist nicht völlig neu. Weil eine Auswertung über den Standard SQL-Parser nicht sehr umfangreich ist und dieser semantischen Fehlern kein sinnvolles Feedback gibt, sind bereits einige Ansätze veröffentlicht worden, die es sich zum Ziel gemacht haben eine SQL-Anfrage näher zu analysieren. Verschiedene Projekte beschäftigen sich dabei \mbox{z. B.} mit dem Aufdecken von semantischen Fehlern. Andere Plattformen konzentrieren sich auf den Lernerfolg, den der Student erreichen soll, und analysieren die Art der Fehler des Lernenden. Damit erreicht man eine Zuteilung von passenderen Aufgaben, sodass der Lernende weder gelangweilt noch überfordert ist. %matchamking

In diesem Abschnitt möchten wir die bereits existierenden Ansätze auf diesem Gebiet betrachten, um dann diese Arbeit einordnen zu können.

\section{SQL-Tutor}

In \cite{sqltut1} beschreibt Antonija Mitrovic ein Lernsystem, das SQL-Tutor genannt wird. Nach Auswahl einer Schwierigkeitsstufe wird dem Studenten ein Datenbankschema und eine Textaufgabe vorgelegt. Der Student hat nun ein Webformular vor sich, in dem sich für jeden Teil der SQL-Anfrage ein Eingabefeld befindet. So werden die Anteile \verb|SELECT|, \verb|FROM|, \verb|WHERE|, \verb|ORDER BY|, \verb|GROUP BY| sowie \verb|HAVING| einzeln eingetragen.

Der SQL-Tutor analysiert nun die Anfrage des Studenten und gibt spezifisches Feedback. Dabei wird nicht nur geklärt, ob die Anfrage korrekt ist, sondern auch, was bei einer falschen Eingabe genau fehlerhaft ist. Das reicht von konkreten Hinweisen auf den spezifischen Teil der Anfrage bis hin zu eindeutigen Hinweisen wie >>Musterlösung enthält einen numerischen Vergleich mit der Spalte a, ihre Lösung enthält aber keinen solchen Vergleich<<.

Umgesetzt wird dieses Programm durch 199 fest einprogrammierte Constraints. Dadurch ist es potentiell möglich bis zu 199 spezifische Hinweismeldungen für den Studenten bereitzustellen. Das reicht von syntaktischen Analysen wie >>Der SELECT-Teil von einer Lösung darf nicht leer sein.<< bis hin zu semantischen Analysen, gepaart mit Wissen über die Domain (Datenbankschema und Musterlösung), bei denen die Lösung des Studenten mit der Musterlösung und dem Datenbankschema verglichen wird. Insbesondere versucht der SQL-Tutor Konstrukte wie numerische Vergleiche mit gewissen Operatoren in der Lösung des Studenten zu finden, wenn diese in der Musterlösung auftauchen. Auch komplexere Constraints, die sicherstellen, dass bei einem numerischen Vergleich \verb|a > 1| das gleiche ist wie \verb| a >= 0| sind vorhanden. 

Allerdings gibt es auch hier Schwächen. Da der verwendete Algorithmus die Constraints nach einander abarbeitet, kann es zu unnötigen Analysen der Anfrage kommen und damit auch zu einem unnötigen Zeitaufwand. Nach eigenen Tests werden manche äquivalente Bedingungen nicht erkannt. So wird \verb|a < 0| für richtig, aber \verb|0 > a| für falsch gehalten. Ähnlich verhält es sich, falls eine der Argumente des Vergleichs das Ergebnis einer Unterabfrage ist. Die Constraints sind fest einprogrammiert und nicht von der Anfrage abhängig und damit genügt es für eine neue Aufgabe Text und Musterlösung einzulesen. 

Der SQL-Tutor lässt außerdem den eingesendeten Lösungsvorschlag auf einer SQL-Datenbank mit Testdaten laufen und vergleicht die Tupel mit den Antworttupeln, die man mit der gespeicherten Musterlösung erhält.

\subsection*{Abgrenzung zum SQL-Tutor}

Der Grundgedanke des SQL-Tutors überschneidet sich durchaus mit dem Ansatz dieser Arbeit. Ein Grundpfeiler des SQL-Tutors ist es, dem Studenten detailliertes Feedback über seine semantischen und syntaktischen Fehler zu geben. Das Programm was im Zuge dieser Arbeit entsteht soll weniger semantische Fehler analysieren, als viel mehr versuchen zwei SQL-Anfragen zu vergleichen, unabhängig davon wie sie aufgeschrieben sind. Des Weiteren bedient sich der SQL-Tutor einer Testdatenbank mit realen Testdaten. Unser Programm soll nur das Datenbankschema kennen und ohne Daten bestimmen, ob zwei Anfragen das gleiche Ergebnis liefern. Erst in einem optionalen zweiten Schritt prüfen wir in unserem Programm, ob beide Anfragen gleiche Ergebnisse auf konkreten Daten liefern. Hat bereits der erster Schritt ein positives Ergebnis gemeldet, dann ist der zweite Schritt unnötig. Der SQL-Tutor operiert allerdings zur Ermittlung der Übereinstimmung nur auf Testdaten. Bei ungünstig gewählten Testdaten kann es passieren, dass der Eindruck entsteht zwei Anfragen wären gleich, obwohl sie es nicht sind. Entstehen kann dies, weil auf den vorhanden Testdaten zufällig beide Anfragen die gleichen Ergebnisse liefern könnten.

\section{SQL-Exploratorium}

Im Artikel \cite{explora1} werden SQL-Lernplattformen in zwei Kategorien eingeteilt. Die erste Kategorie zeichnet sich durch existieren Plattformen aus, welche durch Multimedia versuchen dem Lernenden einzelne Bestandteile der Sprache SQL bildlich darzustellen. Hierfür werden meist Websites mit Multimediainhalten erstellt. Die zweite Kategorie beinhaltet Software, welche die Lösung eines Lernenden analysiert und konkrete Hinweismeldungen gibt. Dazu zählt auch der eben beschriebene SQL-Tutor.

Das SQL-Exploratorium macht es sich nun zur Aufgabe die beiden Ansätze zu verbinden und stellt sich dabei hauptsächlich verwaltungstechnische Fragen wie \mbox{z. B.} 

\begin{itemize}
\item Wie ermögliche ich dem Studenten Zugriff auf verschiedene Lernsysteme, ohne sich mehrfach einloggen zu müssen?
\item Wie können Lernerfolge in einem System einem anderen nutzbar gemacht werden?
\item Wie kann man aus mehreren Logfiles der eingereichten Lösungen eines Studenten, von unterschiedlichen Systemen, einen Wissensstand des Studenten ableiten?
\end{itemize}

Da diese Fragen wenig Relevanz für diese Arbeit haben, betrachten wir im Folgenden welche einzelnen Plattformen für das SQL-Exploratorium genutzt werden.

\subsection{Interactive Examples}

Über eine Schnittstelle, die sich WebEX nennt, hat der Student Zugriff auf insgesamt 64 Beispielanfragen. Wählt man eine Anfrage aus können Teile davon in einer Detailansicht geöffnet werden. Dem Studenten wird dann ausführlich erklärt, was die einzelnen Teile der Anfrage genau bewirken. Sowohl die Beispielanfragen, als auch die Hinweise sind manuell erzeugt und abgespeichert. Hier wird nichts automatisch generiert, daher ist dieses Projekt nicht relevant für die Arbeit. Der Lernerfolg des Studenten wird hier über die ein >>click-log<< geführt, das bedeutet es wird aufgezeichnet, was der Student wann und in welcher Reihenfolge angeklickt hat. So ist es \mbox{z. B.} möglich herauszufinden welche Teile einer bestimmten Anfrage besonders interessant für den Lernenden sind.

\subsection{SQL Knowledge Tester}

Der SQL Knowledge Tester, im Nachfolgendem SQL-KnoT genannt, konzentriert sich darauf Anfragen eines Studenten zu analysieren. Dabei wird dem Studenten zur Laufzeit eine Frage generiert. Dabei werden vorhandene Datenbankschemata in einer bestimmten Art und Weise verknüpft und Testdaten sowie eine Frage für den Studenten generiert. Dies geschieht mit fest einprogrammierten 50 Templates, die in der Lage sind über 400 Fragen zu erzeugen. Zu jeder Frage werden zur Laufzeit Testdaten für die relevanten Datenbanken erzeugt. Ausgewertet wird die Anfrage des Studenten dann, indem die zurückgelieferten Tupel der Studentenanfrage verglichen werden mit den Tupeln, welche die Musterlösung erzeugt. 

\subsection*{Abgrenzung zur Arbeit}

Erwähnenswert ist, dass initial keine Daten existieren. Wie beim Ansatz dieser Arbeit existieren zunächst nur Datenbankschemata. Die Daten und auch die Aufgabe an den Studenten werden aus Templates generiert. Die Auswertung erfolgt dann allerdings durch den Vergleich der zurückgelieferten Tupel der Muster- und Studentenanfrage. Hierbei kann wieder das Problem auftreten, dass für beide Anfragen die erzeugten Testdaten die gleichen Tupel zurückliefern, es aber bei einem anderen Zustand sein kann, dass sich die Tupelmengen unterscheiden. 

Der Ansatz vom SQL-KnoT ist durchaus interessant, wird aber in dieser Arbeit nicht weiter ausgeführt, da wir in dieser Arbeit keine Testdaten erzeugen möchten. Wir benutzen vielmehr Beispieldaten als optionalen zweiten Schritt.

\subsection{Weiteres}

\subsubsection{Adaptive Navigation for SQL Questions}

Hierbei handelt es sich nur um ein Tool, was aufgrund früherer Antworten des Studenten, diesem möglichst passende neue Fragen vorlegen soll. Dieser Teil des SQL-Exploratoriums dient also dazu, den Wissensstand des Studenten festzustellen und ist für diese Arbeit daher unerheblich. 

\section{WIN-RBDI}

Das Programm WINRBDI, welches in \cite{winrbdi1} beschrieben wird verfolgt einen weiteren, interessanten Ansatz. Anstelle von fest vorgegebenen Demoanfragen, wird die eingegebene Anfrage zunächst in esql eingebettet. Die Ausführung der Anfrage wird dann schrittweise durchgeführt. Der Student hat also die Möglichkeit die Anfrage im Schrittmodus, ähnlich eines Debugger, oder im Fortsetzen-Modus auszuführen. Im Schrittmodus wird jeder Teilschritt der Abarbeitung der Anfrage aufgezeigt. Es werden temporär erzeugte Tabellen angegeben sowie auch eine Erklärung welcher Teil der Anfrage für den aktuellen Abarbeitungsschritt verantwortlich ist. So soll es dem Studenten möglich sein, die unmittelbaren Konsequenzen seiner SQL-Anfrage für die Abarbeitung zu begreifen. 

Des Weiteren hilft dieser Ansatz dem Studenten die Abarbeitung einer Anfrage zu Visualisieren, indem die von der WHERE-Klausel betroffene Spalten markiert werden. Dies hilft gerade Lernanfängern bei der Visualisierung von Konzepten wie JOINs.

\subsection*{Abgrenzung zur Arbeit}

Dieser Ansatz hebt sich von den bisherig betrachteten Ansätzen ab. Hier wird dem Studenten durch eine Visualisierung der Ausführung der Anfrage versucht deutlich zu machen, welche Teile der formulierten Anfrage was genau bewirken. Für den Lernerfolg des Studenten ist dies sicherlich hilfreich, zumal eine Visualisierung stets hilft, Zusammenhänge leichter zu begreifen. Diese Arbeit verfolgt allerdings ein anderes Ziel, da sie zwei SQL-Anfragen miteinander vergleicht und nicht versucht die Abarbeitung einer Anfrage verständlicher zu machen.

\section{SQLLint}

>>SQLLint<<, ein Semantik-Prüfer für SQL-Anfragen, beschäftigt sich mit semantischen Fehlern in SQL-Anweisungen, welche unabhängig vom Datenbankzustand auftreten. Dabei behandelt das Projekt Anfragen, von denen man ohne Kenntnis der Aufgabenstellung sagen kann, dass sie, in der vorliegenden Form, nicht beabsichtigt sind. Dies ist wahrscheinlich, wenn man \mbox{z. B.} Teile aus der Anfrage herausstreichen kann ohne die Funktion der Anfrage zu verändern.  Das Problem besteht darin, dass aktuelle DBMS-Systeme solche Anweisungen ohne Fehler- oder Warnmeldung ausführen. Der Nutzer, also insbesondere der lernende Nutzer, ist somit kaum in der Lage überhaupt zu bemerken, dass es einen Fehler in seiner Anfrage gab. Die allgemeine Frage, ob eine Anfrage semantische Fehler enthält ist nicht entscheidbar. Dennoch macht es sich SQLLint zur Aufgabe eine große, typische Teilmenge von SQL-Anfragen zu bearbeiten. Ziel des Projektes ist es, mit semantischen Warn- und Fehlermeldungen, die Codeentwicklung zu beschleunigen und die Anzahl der Fehler darin zu verringern.

Eine wichtige Zielstellung des Projektes ist es, solche Fehlermeldungen in der Lehre einzusetzen. In \cite{sqllint1} wird auch deutlich gemacht, das eine Motivation dieses Projektes aus typischen Fehlern von Studenten entspringt. So wurde im selben Artikel aufgeführt, dass semantische Fehler bei Lernenden am häufigsten auftreten. Unter den drei häufigsten semantischen Fehlern befinden sich: fehlende JOIN Bedingung, (zu) viele Duplikate, unnötiger JOIN. Diese Fehler machen bereits ca. 37 Prozent der semantischen Fehler aus.

Weiterhin fällt auf, dass die Anzahl syntaktischer Fehler mit fortschreitendem Schwierigkeitsgrad der SQL-Anfrage steigen, aber die Anzahl semantischer Fehler nahezu unabhängig von jenem Schwierigkeitsgrad ist. Einfache Anfragen haben sogar zwei mal mehr semantische Fehler als syntaktische Fehler. Siehe dazu Abbildung 4 in \cite{sqllint1}.

\subsection{Algorithmus zum Finden von inkonsistenten Bedingungen}

Die Algorithmen im SQLLint-Projekt sollen inkonsistente Bedingungen finden. Dieses Problem ist allerdings im Allgemeinen unentscheidbar. Dennoch ist es möglich Teilmengen von Anfragen anzugeben, für die man die Konsistenz algorithmisch Entscheiden kann. Folgende Ausführungen zum Algorithmus entstammen der Arbeit >>Proving the Safety of SQL Queries<< von Stefan Brass und Christian Goldberg \cite{brass1}.

Konsistenz im diesem Sinne soll bedeuten, dass es ein endliches Modell (relationaler Datenbankzustand, manchmal auch Datenbankinstanz genannt) existiert, sodass das Ergebnis der Anfrage nicht leer ist.

Wir nehmen im Folgenden an, dass die SQL-Anfragen keine Datentyp Operationen enthalten. Alle atomaren Formeln haben also die Form $t_1\theta t_2$ mit $\theta\in \{=,<>,<,<=,>,>=\}$ und $t_1,t_2$ sind Attribute oder Konstanten. Aggregationsfunktionen sind noch Bestandteil der Forschung und werden daher nicht behandelt.

\subsection{Bedingungen ohne Unteranfragen}

WHERE-Bedingungen, die keine Unteranfrage enthalten, können mit bestimmten Methoden entschieden werden. Ein Beispiel dafür sind die Algorithmen von Guo, Sun und Weiss \cite{decideable1}. Als erster Schritt wird die Negation \verb|NOT| so weit wie möglich an die atomaren Formeln weitergereicht, indem die \verb|DE-MORGAN|'sche Regeln angewendet werden. Dadurch drehen sich die Vergleichsoperatoren um, wir sprechen hierbei von dem >>gegenteiligem Operator<<. Die Menge $O=\{ \{\leq,>\} , \{\geq,<\} , \{ =, =\}, \{\neq,\neq\} \}$ enthält jeweils 2er Mengen von einem Operator und seinem >>gegenteiligem Operator<<. Im nächsten Schritt wird die Bedingung in die disjunktive Normalform (DNF) umgeformt, so dass folgende Struktur entsteht: $\phi_1 \vee ... \vee \phi_n$. Diese ist genau dann konsistent, wenn mindestens ein $\phi_i$ konsistent ist. Nun können wir die Methoden aus \cite{decideable1} anwenden. Im wesentlichen handelt es sich dabei um einen gerichteten Graphen, in dem Knoten markiert sind mit >>Tupelvariable.Attribut<< und Kanten mit $<$ oder $\leq$. Dann werden Intervalle von möglichen Werten für jeden Knoten berechnet. Dabei ist zu beachten, dass die SQL-Datentypen, wie \verb|NUMERIC(1)|, das Intervall zusätzlich einschränken.
Wenn es endlich viele mögliche Werte für einen Knoten gibt, dann können Ungleich-Bedingungen ($t_1<>t_2$) zwischen Knoten wichtig werden und ein Graphfärbungsproblem kodieren. Daher erwarten wir keinen effizienten Algorithmus, wenn es viele $<>$ Bedingungen gibt. In allen anderen Fällen ist die Methode in \cite{decideable1} schnell. Anzumerken ist allerdings noch, dass die Umwandlung in DNF zu exponentiellem Wachstum in der Größe der Anfrage führen kann.

\subsection{Unteranfragen}

Um unnötige Betrachtungen zu vermeiden, beschäftigt sich das SQLLint-Projekt nur mit \verb|EXISTS|-Unteranfragen. Alle anderen Unteranfragen (\verb|IN,>=ALL,| etc.) können auf die \verb|EXISTS|-Unteranfrage reduziert werden. Oracle führt solche Umwandlungen durch, bevor der Optimierer beginnt an der Anfrage zu arbeiten.

Die Idee zur Behandlung von Unteranfragen stammt aus bekannten Methoden der automatischen Beweiser. Hierzu wird in der Arbeit \cite{brass1} eine Variante der Skolemisierung vorgestellt. Das genaue Vorgehen wird in in jenem Artikel erklärt.

\subsection{Unnötige logische Komplikationen}

Es kann vorkommen, dass eine Teilbedingung inkonsistent ist, die gesamte Bedingung allerdings dennoch konsistent ist (Aufgrund der Disjunktion). Ebenso denkbar ist der umgekehrte Fall, dass also Unterbedingungen Tautologien sind. Beide Vorkommnisse sind vermutlich nicht gewollt und können zu einem unerwünschte Verhalten einer Anfrage führen. Wie in \cite{brass2} festgestellt wurde, werden in Klausuren von Studenten auch öfter unnötige Bedingungen angegeben, welche bereits per Definition impliziert werden. Als Beispiel betrachten wir die Bedingung \verb|A IS NOT NULL|. Diese wird unnötig, wenn wir wissen, dass \verb|A| bereits als \verb|NOT NULL| definiert ist.

Im Folgenden wird in \cite{brass2} eine mögliche Formalisierung der Voraussetzung für >>keine unnötigen logischen Komplikationen<< erläutert. Immer wenn in der DNF der Anfragebedingung eine Unterbedingung mit >>true<< oder >>false<< ersetzt wird, ist das Ergebnis nicht zur Ausgangsbedingung äquivalent.

Realisiert wird dies durch eine Reihe von Konsistenzprüfungen. Es sei die DNF der Anfragebedingung $C_1\vee ...\vee C_m$, mit $C_i=(A_{i,1}\wedge ...\wedge A_{i,n_i})$. Unser Kriterium ist genau dann erfüllt, wenn die folgenden Formeln alle konsistent sind:

\begin{enumerate}
\item $\neg(C_1 \vee ... \vee C_m)$ - Die Negation der gesamten Formel. Ansonsten könnte man diese durch >>true<< ersetzen.
\item $C_1 \wedge \neg(C_1 \vee ... \vee C_{i-1} \vee C_{i+1} \vee ... \vee C_m)$ mit $i\in [1,m]\cap \mathbb{N}$. Ansonsten könnte $C_i$ mit >>false<< ersetzt werden.
\item  $A_{i,1} \wedge ... \wedge A_{i,j-1} \wedge  \neg A_{i,j} \wedge A_{i,j+1} \wedge ... \wedge A_{i,n_i} \wedge \neg(C_1 \vee ... \vee C_{i-1} \vee C_{i+1} \vee ... \vee C_m)$ mit $i\in [1,m] \cap \mathbb{N}$, $j\in [1,n_i]\cap \mathbb{N}$. Ansonsten könnte $A_{i,j}$ mit >>true<< ersetzt werden.
\end{enumerate}

Zu weiteren unnötigen logischen Komplikationen zählen zu allgemeine Vergleichsoperatoren ($>=$ anstelle von $=$). Weiterhin gehören unnötige JOINs zu einem wichtigen Typ von unnötigen logischen Komplikationen.

\subsection{Laufzeitfehler}

Als Bemerkung ist festzuhalten, dass sich das SQLLint-Projekt auch mit Laufzeitfehlern beschäftigt. Als Beispiel stelle man sich folgende SQL-Bedingung vor: \verb|A=(SELECT ...)| Es muss hier sichergestellt werden, dass die \verb|SELECT|-Unteranfrage nur einen Rückgabewert hat. Solche Fehler sind schwierig zu finden, da sie nicht immer Auftreten müssen. 

Wie das SQLLint-Projekt damit umgeht, soll hier nicht weiter besprochen werden. Details dazu sind zu finden in \cite{brass2}. 

\subsection*{Zusammenhang zu dieser Arbeit}

Obwohl SQLLint auf den ersten Blick eine andere Zielstellung als diese Arbeit verfolgt, so sind doch einige Ansätze deckungsgleich. Einige der Ansätze von SQLLint können Grundlagen für diese Arbeit sein. Der Ansatz der Standardisierung der SQL-Anfragen ist mit Umwandeln der Formeln in eine Normalform ein guter Ansatzpunkt und zeigt wie sich komplexe Bedingungen vereinheitlichen lassen. Auch die Erkenntnis, dass sich alle Unteranfragen auf \verb|EXISTS| Unteranfragen reduzieren lassen, wird helfen die Unteranfragen zu standardisieren. Dadurch wird die Vielfalt der Unteranfragen eingeschränkt und ein Vergleich zweier SQL-Anfragen wird vereinfacht.

Weiterhin könnte man in späteren Ausbaustadien des Programmes, welches im Rahmen dieser Arbeit entsteht, die Funktionalitäten des SQLLint einbeziehen . Dies würde die Art des Feedbacks für den Lernenden deutlich verbessern, da wir uns in dieser Arbeit zunächst auf das Vergleichen von zwei SQL-Anfragen konzentrieren. Dabei stehen vor allem Hinweise im Vordergrund, die dem Lernenden zeigen sollen, warum seine Lösung mit der Musterlösung noch nicht übereinstimmen kann.

Ein davon unabhängiges Feedback für die Anfrage des Lernenden würde den Lernverlauf stark beschleunigen und mit hoher Wahrscheinlichkeit sogar die Fehler der Anfrage eliminieren, so dass die Anfrage dann auf die Musterlösung passt.